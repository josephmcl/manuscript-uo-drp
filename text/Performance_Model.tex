%
%
%
\noindent
With the decomposed matrices given by Eq. \ref{eqn:finc}--\ref{eqn:fine}, and the analogous form for $\textbf{M}$, we avoid storing identical sub-matrices, which a typical sparse matrix format would not. 
Additionally, where multiple operations utilize the same $\textbf{F}^i$ $\textbf{M}_i$, we only perform this operation once. 
This holds for all sub-matrices of $\textbf{F}^\intercal$ as each $(\textbf{F}^\intercal)^i \equiv (\textbf{F}^i)^\intercal$.
Such operations are found in computing the global, trace problem \emph{i.e.}, Eq. \ref{eqn:global_system_a}--\ref{eqn:global_system_c}.

The benefit of this is now, each local system is computed as a series of smaller problems. 
This decomposition strategy is characteristic of hyrbidized methods, and is the fundamental reason to utilize these methods, in particular for larger problems sizes within a shared-memory context \citep{kozdon2021hybridized, kolev2021efficient, fernandez2017hybridized}. 

\begin{figure}[h]
	\centering
	
\begin{tikzpicture}[scale=1.6]
\filldraw[color=black!10] (0, 0) rectangle (3, 3);
\filldraw[color=black!30, text=black] (0.00cm, 2.66cm) rectangle (0.33cm, 3.00cm) node[pos=.5] {$\textbf{M}_1$};
\filldraw[color=black!30, text=black] (0.33cm, 2.33cm) rectangle (0.66cm, 2.66cm) node[pos=.5] {$\textbf{M}_2$};
\filldraw[color=black!30, text=black] (0.66cm, 2.00cm) rectangle (1.00cm, 2.33cm) node[pos=.5] {$\textbf{M}_3$};
\filldraw[color=black!30, text=black] (1.00cm, 1.66cm) rectangle (1.33cm, 2.00cm) node[pos=.5] {$\textbf{M}_4$};
\filldraw[color=black!30, text=black] (1.33cm, 1.33cm) rectangle (1.66cm, 1.66cm) node[pos=.5] {$\textbf{M}_5$};
\filldraw[color=black!30, text=black] (1.66cm, 1.00cm) rectangle (2.00cm, 1.33cm) node[pos=.5] {$\textbf{M}_6$};
\filldraw[color=black!30, text=black] (2.00cm, 0.66cm) rectangle (2.33cm, 1.00cm) node[pos=.5] {$\textbf{M}_7$};
\filldraw[color=black!30, text=black] (2.33cm, 0.33cm) rectangle (2.66cm, 0.66cm) node[pos=.5] {$\textbf{M}_8$};
\filldraw[color=black!30, text=black] (2.66cm, 0.00cm) rectangle (3.00cm, 0.33cm) node[pos=.5] {$\textbf{M}_9$};

\filldraw[color=black!10,xshift=0.1cm] (3, 0) rectangle (4.33, 3);

\filldraw[color=red!30,xshift=0.1cm] (3.00cm, 2.66cm) rectangle (3.11cm, 3.00cm);
\filldraw[color=red!30,xshift=0.1cm] (3.11cm, 2.33cm) rectangle (3.22cm, 2.66cm);
\filldraw[color=red!30,xshift=0.1cm] (3.22cm, 1.66cm) rectangle (3.33cm, 2.00cm);
\filldraw[color=red!30,xshift=0.1cm] (3.33cm, 1.33cm) rectangle (3.44cm, 1.66cm);
\filldraw[color=red!30,xshift=0.1cm] (3.44cm, 0.66cm) rectangle (3.55cm, 1.00cm);
\filldraw[color=red!30,xshift=0.1cm] (3.55cm, 0.33cm) rectangle (3.66cm, 0.66cm);

\filldraw[color=green!30,xshift=0.1cm] (3.00cm, 2.33cm) rectangle (3.11cm, 2.66cm);
\filldraw[color=green!30,xshift=0.1cm] (3.11cm, 2.00cm) rectangle (3.22cm, 2.33cm);
\filldraw[color=green!30,xshift=0.1cm] (3.22cm, 1.33cm) rectangle (3.33cm, 1.66cm);
\filldraw[color=green!30,xshift=0.1cm] (3.33cm, 1.00cm) rectangle (3.44cm, 1.33cm);
\filldraw[color=green!30,xshift=0.1cm] (3.44cm, 0.33cm) rectangle (3.55cm, 0.66cm);
\filldraw[color=green!30,xshift=0.1cm] (3.55cm, 0.00cm) rectangle (3.66cm, 0.33cm);

\filldraw[color=violet!30,xshift=0.1cm] (3.66cm, 2.66cm) rectangle (3.77cm, 3.00cm);
\filldraw[color=violet!30,xshift=0.1cm] (3.77cm, 2.33cm) rectangle (3.88cm, 2.66cm);
\filldraw[color=violet!30,xshift=0.1cm] (3.88cm, 2.00cm) rectangle (4.00cm, 2.33cm);
\filldraw[color=violet!30,xshift=0.1cm] (4.00cm, 1.66cm) rectangle (4.11cm, 2.00cm);
\filldraw[color=violet!30,xshift=0.1cm] (4.11cm, 1.33cm) rectangle (4.22cm, 1.66cm);
\filldraw[color=violet!30,xshift=0.1cm] (4.22cm, 1.00cm) rectangle (4.33cm, 1.33cm);

\filldraw[color=orange!30,xshift=0.1cm] (3.66cm, 1.66cm) rectangle (3.77cm, 2.00cm);
\filldraw[color=orange!30,xshift=0.1cm] (3.77cm, 1.33cm) rectangle (3.88cm, 1.66cm);
\filldraw[color=orange!30,xshift=0.1cm] (3.88cm, 1.00cm) rectangle (4.00cm, 1.33cm);
\filldraw[color=orange!30,xshift=0.1cm] (4.00cm, 0.66cm) rectangle (4.11cm, 1.00cm);
\filldraw[color=orange!30,xshift=0.1cm] (4.11cm, 0.33cm) rectangle (4.22cm, 0.66cm);
\filldraw[color=orange!30,xshift=0.1cm] (4.22cm, 0.00cm) rectangle (4.33cm, 0.33cm);

\filldraw[color=black!10,yshift=-0.1cm] (0, -1.33) rectangle (3, 0);

\begin{scope}[rotate around={-90:(0,0)},yscale=1,xscale=-1,xshift=-3.86cm,yshift=-0.33cm]
 
\filldraw[color=red!30,xshift=0.1cm] (3.00cm, 2.66cm) rectangle (3.11cm, 3.00cm);
\filldraw[color=red!30,xshift=0.1cm] (3.11cm, 2.33cm) rectangle (3.22cm, 2.66cm);
\filldraw[color=red!30,xshift=0.1cm] (3.22cm, 1.66cm) rectangle (3.33cm, 2.00cm);
\filldraw[color=red!30,xshift=0.1cm] (3.33cm, 1.33cm) rectangle (3.44cm, 1.66cm);
\filldraw[color=red!30,xshift=0.1cm] (3.44cm, 0.66cm) rectangle (3.55cm, 1.00cm);
\filldraw[color=red!30,xshift=0.1cm] (3.55cm, 0.33cm) rectangle (3.66cm, 0.66cm);

\end{scope}
\begin{scope}[rotate around={-90:(0,0)},yscale=1,xscale=-1,xshift=-3.86cm,yshift=0.34cm]

\filldraw[color=green!30,xshift=0.1cm] (3.00cm, 2.33cm) rectangle (3.11cm, 2.66cm);
\filldraw[color=green!30,xshift=0.1cm] (3.11cm, 2.00cm) rectangle (3.22cm, 2.33cm);
\filldraw[color=green!30,xshift=0.1cm] (3.22cm, 1.33cm) rectangle (3.33cm, 1.66cm);
\filldraw[color=green!30,xshift=0.1cm] (3.33cm, 1.00cm) rectangle (3.44cm, 1.33cm);
\filldraw[color=green!30,xshift=0.1cm] (3.44cm, 0.33cm) rectangle (3.55cm, 0.66cm);
\filldraw[color=green!30,xshift=0.1cm] (3.55cm, 0.00cm) rectangle (3.66cm, 0.33cm);

\end{scope}
\begin{scope}[rotate around={-90:(0,0)},yscale=1,xscale=-1,xshift=-5.19cm,yshift=1cm] 

\filldraw[color=orange!30,xshift=0.1cm] (3.66cm, 1.66cm) rectangle (3.77cm, 2.00cm);
\filldraw[color=orange!30,xshift=0.1cm] (3.77cm, 1.33cm) rectangle (3.88cm, 1.66cm);
\filldraw[color=orange!30,xshift=0.1cm] (3.88cm, 1.00cm) rectangle (4.00cm, 1.33cm);
\filldraw[color=orange!30,xshift=0.1cm] (4.00cm, 0.66cm) rectangle (4.11cm, 1.00cm);
\filldraw[color=orange!30,xshift=0.1cm] (4.11cm, 0.33cm) rectangle (4.22cm, 0.66cm);
\filldraw[color=orange!30,xshift=0.1cm] (4.22cm, 0.00cm) rectangle (4.33cm, 0.33cm);

\end{scope}

\begin{scope}[rotate around={-90:(0,0)},yscale=1,xscale=-1,xshift=-5.19cm,yshift=-1cm]

\filldraw[color=violet!30,xshift=0.1cm] (3.66cm, 2.66cm) rectangle (3.77cm, 3.00cm);
\filldraw[color=violet!30,xshift=0.1cm] (3.77cm, 2.33cm) rectangle (3.88cm, 2.66cm);
\filldraw[color=violet!30,xshift=0.1cm] (3.88cm, 2.00cm) rectangle (4.00cm, 2.33cm);
\filldraw[color=violet!30,xshift=0.1cm] (4.00cm, 1.66cm) rectangle (4.11cm, 2.00cm);
\filldraw[color=violet!30,xshift=0.1cm] (4.11cm, 1.33cm) rectangle (4.22cm, 1.66cm);
\filldraw[color=violet!30,xshift=0.1cm] (4.22cm, 1.00cm) rectangle (4.33cm, 1.33cm);

\end{scope}

\filldraw[color=black!10,xshift=0.1cm,yshift=-0.1cm] (3, -1.33) rectangle (4.33, 0);

\draw[color=black!30,xshift=0.1cm,yshift=-0.1cm] (3, 0) -- (4.33, -1.33);

\draw [black, thick] (-0.1,-1.43) to [square left brace] (-0.1,3);

\draw [black, thick] (4.53,-1.43) to [square right brace] (4.53,3);

\filldraw[draw=black,fill=white] (0.39cm, -2.1cm) rectangle (4.09cm, -1.53cm);

\filldraw[color=black!10,xshift=0.49cm, text=black] (0cm, -2cm) rectangle (0.5cm, -1.63cm) node[pos=.5] {$0$};
\filldraw[color=black!30,xshift=1.09cm, text=black] (0cm, -2cm) rectangle (0.5cm, -1.63cm) node[pos=.5] {$\textbf{M}_i$};
\filldraw[color=red!30,xshift=1.69cm, text=black] (0cm, -2cm) rectangle (0.5cm, -1.63cm) node[pos=.5] {$\textbf{F}^1$};
\filldraw[color=green!30,xshift=2.29cm, text=black] (0cm, -2cm) rectangle (0.5cm, -1.63cm) node[pos=.5] {$\textbf{F}^2$};
\filldraw[color=violet!30,xshift=2.89cm, text=black] (0cm, -2cm) rectangle (0.5cm, -1.63cm) node[pos=.5] {$\textbf{F}^3$};
\filldraw[color=orange!30,xshift=3.49cm, text=black] (0cm, -2cm) rectangle (0.5cm, -1.63cm) node[pos=.5] {$\textbf{F}^4$};
%\draw [decorate,decoration={brace,amplitude=10pt},xshift=-2pt,yshift=0pt]
%(0cm,2cm) -- (0cm,3cm) node [black,midway,xshift=-0.6cm] 
%{\footnotesize $n$};

\end{tikzpicture}

	\caption{A visualization of the non-zero pattern of the matrix operator of a 3-by-3 instance of the hybrid problem specified 
	in \eqnref{eqn:hybrid_system}.}
	\label{fig:block_diagram}
\end{figure}



%
%
%
\subsubsection{Complexity of the trace system} 

The hybrid formulation permits us to compute much larger problems with less memory and fewer reads, especially if the problem is mostly homogeneous.
This is of particular interest as the 3 significant computations used in computing this problem \emph{i.e.}, MatMul, SpMV, and linear solve via direct solve, are often memory-bound operations. 
Additionally, we reduce the computational complexity of solving the system by instead solving several smaller systems. 
In both cases this allows us to express 
the equations (\ref{eqn:global_system_a}--\ref{eqn:global_system_c}, 
\ref{eqn:local_system}) as a concatenation of decomposed, smaller problems.

To evaluate this the performance characteristics we fix the global number of volume points, $\bar{n}^2$, and vary the number of elements, to understand if there exists an optimal trade-off between the size of the decomposed matrices and quantity of computations derived from the decomposed matrices.

\begin{figure}[h]
	\centering
	\begin{tikzpicture}[scale=1]
\filldraw[color=black!10] (0, 0) rectangle (1.5, 1.5);
\filldraw[color=black!30, text=black] (0.0cm,1.5cm) rectangle (0.125cm,1.375cm);
\filldraw[color=black!30, text=black] (0.0cm,1.375cm) rectangle (0.125cm,1.25cm);
\filldraw[color=black!30, text=black] (0.0cm,0.75cm) rectangle (0.125cm,0.625cm);
\filldraw[color=black!30, text=black] (0.0cm,0.625cm) rectangle (0.125cm,0.5cm);
\filldraw[color=black!30, text=black] (0.125cm,1.5cm) rectangle (0.25cm,1.375cm);
\filldraw[color=black!30, text=black] (0.125cm,1.375cm) rectangle (0.25cm,1.25cm);
\filldraw[color=black!30, text=black] (0.125cm,0.625cm) rectangle (0.25cm,0.5cm);
\filldraw[color=black!30, text=black] (0.125cm,0.5cm) rectangle (0.25cm,0.375cm);
\filldraw[color=black!30, text=black] (0.25cm,1.25cm) rectangle (0.375cm,1.125cm);
\filldraw[color=black!30, text=black] (0.25cm,1.125cm) rectangle (0.375cm,1.0cm);
\filldraw[color=black!30, text=black] (0.25cm,0.75cm) rectangle (0.375cm,0.625cm);
\filldraw[color=black!30, text=black] (0.25cm,0.625cm) rectangle (0.375cm,0.5cm);
\filldraw[color=black!30, text=black] (0.25cm,0.375cm) rectangle (0.375cm,0.25cm);
\filldraw[color=black!30, text=black] (0.25cm,0.25cm) rectangle (0.375cm,0.125cm);
\filldraw[color=black!30, text=black] (0.375cm,1.25cm) rectangle (0.5cm,1.125cm);
\filldraw[color=black!30, text=black] (0.375cm,1.125cm) rectangle (0.5cm,1.0cm);
\filldraw[color=black!30, text=black] (0.375cm,0.625cm) rectangle (0.5cm,0.5cm);
\filldraw[color=black!30, text=black] (0.375cm,0.5cm) rectangle (0.5cm,0.375cm);
\filldraw[color=black!30, text=black] (0.375cm,0.25cm) rectangle (0.5cm,0.125cm);
\filldraw[color=black!30, text=black] (0.375cm,0.125cm) rectangle (0.5cm,0.0cm);
\filldraw[color=black!30, text=black] (0.5cm,1.0cm) rectangle (0.625cm,0.875cm);
\filldraw[color=black!30, text=black] (0.5cm,0.875cm) rectangle (0.625cm,0.75cm);
\filldraw[color=black!30, text=black] (0.5cm,0.375cm) rectangle (0.625cm,0.25cm);
\filldraw[color=black!30, text=black] (0.5cm,0.25cm) rectangle (0.625cm,0.125cm);
\filldraw[color=black!30, text=black] (0.625cm,1.0cm) rectangle (0.75cm,0.875cm);
\filldraw[color=black!30, text=black] (0.625cm,0.875cm) rectangle (0.75cm,0.75cm);
\filldraw[color=black!30, text=black] (0.625cm,0.25cm) rectangle (0.75cm,0.125cm);
\filldraw[color=black!30, text=black] (0.625cm,0.125cm) rectangle (0.75cm,0.0cm);
\filldraw[color=black!30, text=black] (0.75cm,1.5cm) rectangle (0.875cm,1.375cm);
\filldraw[color=black!30, text=black] (0.75cm,1.25cm) rectangle (0.875cm,1.125cm);
\filldraw[color=black!30, text=black] (0.75cm,0.75cm) rectangle (0.875cm,0.625cm);
\filldraw[color=black!30, text=black] (0.75cm,0.375cm) rectangle (0.875cm,0.25cm);
\filldraw[color=black!30, text=black] (0.875cm,1.5cm) rectangle (1.0cm,1.375cm);
\filldraw[color=black!30, text=black] (0.875cm,1.375cm) rectangle (1.0cm,1.25cm);
\filldraw[color=black!30, text=black] (0.875cm,1.25cm) rectangle (1.0cm,1.125cm);
\filldraw[color=black!30, text=black] (0.875cm,1.125cm) rectangle (1.0cm,1.0cm);
\filldraw[color=black!30, text=black] (0.875cm,0.625cm) rectangle (1.0cm,0.5cm);
\filldraw[color=black!30, text=black] (0.875cm,0.25cm) rectangle (1.0cm,0.125cm);
\filldraw[color=black!30, text=black] (1.0cm,1.375cm) rectangle (1.125cm,1.25cm);
\filldraw[color=black!30, text=black] (1.0cm,1.125cm) rectangle (1.125cm,1.0cm);
\filldraw[color=black!30, text=black] (1.0cm,0.5cm) rectangle (1.125cm,0.375cm);
\filldraw[color=black!30, text=black] (1.0cm,0.125cm) rectangle (1.125cm,0.0cm);
\filldraw[color=black!30, text=black] (1.125cm,1.25cm) rectangle (1.25cm,1.125cm);
\filldraw[color=black!30, text=black] (1.125cm,1.0cm) rectangle (1.25cm,0.875cm);
\filldraw[color=black!30, text=black] (1.125cm,0.75cm) rectangle (1.25cm,0.625cm);
\filldraw[color=black!30, text=black] (1.125cm,0.375cm) rectangle (1.25cm,0.25cm);
\filldraw[color=black!30, text=black] (1.25cm,1.25cm) rectangle (1.375cm,1.125cm);
\filldraw[color=black!30, text=black] (1.25cm,1.125cm) rectangle (1.375cm,1.0cm);
\filldraw[color=black!30, text=black] (1.25cm,1.0cm) rectangle (1.375cm,0.875cm);
\filldraw[color=black!30, text=black] (1.25cm,0.875cm) rectangle (1.375cm,0.75cm);
\filldraw[color=black!30, text=black] (1.25cm,0.625cm) rectangle (1.375cm,0.5cm);
\filldraw[color=black!30, text=black] (1.25cm,0.25cm) rectangle (1.375cm,0.125cm);
\filldraw[color=black!30, text=black] (1.375cm,1.125cm) rectangle (1.5cm,1.0cm);
\filldraw[color=black!30, text=black] (1.375cm,0.875cm) rectangle (1.5cm,0.75cm);
\filldraw[color=black!30, text=black] (1.375cm,0.5cm) rectangle (1.5cm,0.375cm);
\filldraw[color=black!30, text=black] (1.375cm,0.125cm) rectangle (1.5cm,0.0cm);
\draw [black, thick] (-0.2,0) to [square left brace] (-0.2,1.5);
\draw [black, thick] (1.7,0) to [square right brace] (1.7, 1.5);
\node[draw=none,align=center, anchor=north] at (0.75,-0.1) {\footnotesize $\symbf{\lambda}_\textbf{A} \in \mathbb{R}^{3\,360 \times 3\,360}$ \\ \footnotesize  $\bar{n} = 840$, $\ell^2 = 9$, \\ \footnotesize $\text{nnz} = 5.3\mathrm{e}6$};

\end{tikzpicture}
\hspace{1cm}
\begin{tikzpicture}
\draw [black, thick] (-0.2,0) to [square left brace] (-0.2,3);
\draw [black, thick] (3.2,0) to [square right brace] (3.2, 3);
\filldraw[color=black!10] (0, 0) rectangle (3, 3);
\filldraw[color=black!30, text=black] (0.0cm,3.0cm) rectangle (0.07501875468867217cm,2.924981245311328cm);
\filldraw[color=black!30, text=black] (0.0cm,2.924981245311328cm) rectangle (0.07501875468867217cm,2.8499624906226555cm);
\filldraw[color=black!30, text=black] (0.0cm,1.4996249062265568cm) rectangle (0.07501875468867217cm,1.4246061515378845cm);
\filldraw[color=black!30, text=black] (0.0cm,1.4246061515378845cm) rectangle (0.07501875468867217cm,1.3495873968492123cm);
\filldraw[color=black!30, text=black] (0.07501875468867217cm,3.0cm) rectangle (0.15003750937734434cm,2.924981245311328cm);
\filldraw[color=black!30, text=black] (0.07501875468867217cm,2.924981245311328cm) rectangle (0.15003750937734434cm,2.8499624906226555cm);
\filldraw[color=black!30, text=black] (0.07501875468867217cm,2.8499624906226555cm) rectangle (0.15003750937734434cm,2.7749437359339835cm);
\filldraw[color=black!30, text=black] (0.07501875468867217cm,1.4246061515378845cm) rectangle (0.15003750937734434cm,1.3495873968492123cm);
\filldraw[color=black!30, text=black] (0.07501875468867217cm,1.3495873968492123cm) rectangle (0.15003750937734434cm,1.2745686421605402cm);
\filldraw[color=black!30, text=black] (0.15003750937734434cm,2.924981245311328cm) rectangle (0.2250562640660165cm,2.8499624906226555cm);
\filldraw[color=black!30, text=black] (0.15003750937734434cm,2.8499624906226555cm) rectangle (0.2250562640660165cm,2.7749437359339835cm);
\filldraw[color=black!30, text=black] (0.15003750937734434cm,2.7749437359339835cm) rectangle (0.2250562640660165cm,2.6999249812453114cm);
\filldraw[color=black!30, text=black] (0.15003750937734434cm,1.3495873968492123cm) rectangle (0.2250562640660165cm,1.2745686421605402cm);
\filldraw[color=black!30, text=black] (0.15003750937734434cm,1.2745686421605402cm) rectangle (0.2250562640660165cm,1.199549887471868cm);
\filldraw[color=black!30, text=black] (0.2250562640660165cm,2.8499624906226555cm) rectangle (0.30007501875468867cm,2.7749437359339835cm);
\filldraw[color=black!30, text=black] (0.2250562640660165cm,2.7749437359339835cm) rectangle (0.30007501875468867cm,2.6999249812453114cm);
\filldraw[color=black!30, text=black] (0.2250562640660165cm,1.2745686421605402cm) rectangle (0.30007501875468867cm,1.199549887471868cm);
\filldraw[color=black!30, text=black] (0.2250562640660165cm,1.199549887471868cm) rectangle (0.30007501875468867cm,1.1245311327831957cm);
\filldraw[color=black!30, text=black] (0.30007501875468867cm,2.6999249812453114cm) rectangle (0.3750937734433608cm,2.6249062265566394cm);
\filldraw[color=black!30, text=black] (0.30007501875468867cm,2.6249062265566394cm) rectangle (0.3750937734433608cm,2.549887471867967cm);
\filldraw[color=black!30, text=black] (0.30007501875468867cm,1.4996249062265568cm) rectangle (0.3750937734433608cm,1.4246061515378845cm);
\filldraw[color=black!30, text=black] (0.30007501875468867cm,1.4246061515378845cm) rectangle (0.3750937734433608cm,1.3495873968492123cm);
\filldraw[color=black!30, text=black] (0.30007501875468867cm,1.1245311327831957cm) rectangle (0.3750937734433608cm,1.0495123780945237cm);
\filldraw[color=black!30, text=black] (0.30007501875468867cm,1.0495123780945237cm) rectangle (0.3750937734433608cm,0.9744936234058517cm);
\filldraw[color=black!30, text=black] (0.3750937734433608cm,2.6999249812453114cm) rectangle (0.450112528132033cm,2.6249062265566394cm);
\filldraw[color=black!30, text=black] (0.3750937734433608cm,2.6249062265566394cm) rectangle (0.450112528132033cm,2.549887471867967cm);
\filldraw[color=black!30, text=black] (0.3750937734433608cm,2.549887471867967cm) rectangle (0.450112528132033cm,2.474868717179295cm);
\filldraw[color=black!30, text=black] (0.3750937734433608cm,1.4246061515378845cm) rectangle (0.450112528132033cm,1.3495873968492123cm);
\filldraw[color=black!30, text=black] (0.3750937734433608cm,1.3495873968492123cm) rectangle (0.450112528132033cm,1.2745686421605402cm);
\filldraw[color=black!30, text=black] (0.3750937734433608cm,1.0495123780945237cm) rectangle (0.450112528132033cm,0.9744936234058517cm);
\filldraw[color=black!30, text=black] (0.3750937734433608cm,0.9744936234058517cm) rectangle (0.450112528132033cm,0.8994748687171792cm);
\filldraw[color=black!30, text=black] (0.450112528132033cm,2.6249062265566394cm) rectangle (0.5251312828207052cm,2.549887471867967cm);
\filldraw[color=black!30, text=black] (0.450112528132033cm,2.549887471867967cm) rectangle (0.5251312828207052cm,2.474868717179295cm);
\filldraw[color=black!30, text=black] (0.450112528132033cm,2.474868717179295cm) rectangle (0.5251312828207052cm,2.399849962490623cm);
\filldraw[color=black!30, text=black] (0.450112528132033cm,1.3495873968492123cm) rectangle (0.5251312828207052cm,1.2745686421605402cm);
\filldraw[color=black!30, text=black] (0.450112528132033cm,1.2745686421605402cm) rectangle (0.5251312828207052cm,1.199549887471868cm);
\filldraw[color=black!30, text=black] (0.450112528132033cm,0.9744936234058517cm) rectangle (0.5251312828207052cm,0.8994748687171792cm);
\filldraw[color=black!30, text=black] (0.450112528132033cm,0.8994748687171792cm) rectangle (0.5251312828207052cm,0.8244561140285072cm);
\filldraw[color=black!30, text=black] (0.5251312828207052cm,2.549887471867967cm) rectangle (0.6001500375093773cm,2.474868717179295cm);
\filldraw[color=black!30, text=black] (0.5251312828207052cm,2.474868717179295cm) rectangle (0.6001500375093773cm,2.399849962490623cm);
\filldraw[color=black!30, text=black] (0.5251312828207052cm,1.2745686421605402cm) rectangle (0.6001500375093773cm,1.199549887471868cm);
\filldraw[color=black!30, text=black] (0.5251312828207052cm,1.199549887471868cm) rectangle (0.6001500375093773cm,1.1245311327831957cm);
\filldraw[color=black!30, text=black] (0.5251312828207052cm,0.8994748687171792cm) rectangle (0.6001500375093773cm,0.8244561140285072cm);
\filldraw[color=black!30, text=black] (0.5251312828207052cm,0.8244561140285072cm) rectangle (0.6001500375093773cm,0.7494373593398351cm);
\filldraw[color=black!30, text=black] (0.6001500375093773cm,2.399849962490623cm) rectangle (0.6751687921980495cm,2.3248312078019504cm);
\filldraw[color=black!30, text=black] (0.6001500375093773cm,2.3248312078019504cm) rectangle (0.6751687921980495cm,2.2498124531132784cm);
\filldraw[color=black!30, text=black] (0.6001500375093773cm,1.1245311327831957cm) rectangle (0.6751687921980495cm,1.0495123780945237cm);
\filldraw[color=black!30, text=black] (0.6001500375093773cm,1.0495123780945237cm) rectangle (0.6751687921980495cm,0.9744936234058517cm);
\filldraw[color=black!30, text=black] (0.6001500375093773cm,0.7494373593398351cm) rectangle (0.6751687921980495cm,0.6744186046511627cm);
\filldraw[color=black!30, text=black] (0.6001500375093773cm,0.6744186046511627cm) rectangle (0.6751687921980495cm,0.5993998499624906cm);
\filldraw[color=black!30, text=black] (0.6751687921980495cm,2.399849962490623cm) rectangle (0.7501875468867216cm,2.3248312078019504cm);
\filldraw[color=black!30, text=black] (0.6751687921980495cm,2.3248312078019504cm) rectangle (0.7501875468867216cm,2.2498124531132784cm);
\filldraw[color=black!30, text=black] (0.6751687921980495cm,2.2498124531132784cm) rectangle (0.7501875468867216cm,2.1747936984246063cm);
\filldraw[color=black!30, text=black] (0.6751687921980495cm,1.0495123780945237cm) rectangle (0.7501875468867216cm,0.9744936234058517cm);
\filldraw[color=black!30, text=black] (0.6751687921980495cm,0.9744936234058517cm) rectangle (0.7501875468867216cm,0.8994748687171792cm);
\filldraw[color=black!30, text=black] (0.6751687921980495cm,0.6744186046511627cm) rectangle (0.7501875468867216cm,0.5993998499624906cm);
\filldraw[color=black!30, text=black] (0.6751687921980495cm,0.5993998499624906cm) rectangle (0.7501875468867216cm,0.5243810952738186cm);
\filldraw[color=black!30, text=black] (0.7501875468867216cm,2.3248312078019504cm) rectangle (0.8252063015753939cm,2.2498124531132784cm);
\filldraw[color=black!30, text=black] (0.7501875468867216cm,2.2498124531132784cm) rectangle (0.8252063015753939cm,2.1747936984246063cm);
\filldraw[color=black!30, text=black] (0.7501875468867216cm,2.1747936984246063cm) rectangle (0.8252063015753939cm,2.099774943735934cm);
\filldraw[color=black!30, text=black] (0.7501875468867216cm,0.9744936234058517cm) rectangle (0.8252063015753939cm,0.8994748687171792cm);
\filldraw[color=black!30, text=black] (0.7501875468867216cm,0.8994748687171792cm) rectangle (0.8252063015753939cm,0.8244561140285072cm);
\filldraw[color=black!30, text=black] (0.7501875468867216cm,0.5993998499624906cm) rectangle (0.8252063015753939cm,0.5243810952738186cm);
\filldraw[color=black!30, text=black] (0.7501875468867216cm,0.5243810952738186cm) rectangle (0.8252063015753939cm,0.4493623405851461cm);
\filldraw[color=black!30, text=black] (0.8252063015753939cm,2.2498124531132784cm) rectangle (0.900225056264066cm,2.1747936984246063cm);
\filldraw[color=black!30, text=black] (0.8252063015753939cm,2.1747936984246063cm) rectangle (0.900225056264066cm,2.099774943735934cm);
\filldraw[color=black!30, text=black] (0.8252063015753939cm,0.8994748687171792cm) rectangle (0.900225056264066cm,0.8244561140285072cm);
\filldraw[color=black!30, text=black] (0.8252063015753939cm,0.8244561140285072cm) rectangle (0.900225056264066cm,0.7494373593398351cm);
\filldraw[color=black!30, text=black] (0.8252063015753939cm,0.5243810952738186cm) rectangle (0.900225056264066cm,0.4493623405851461cm);
\filldraw[color=black!30, text=black] (0.8252063015753939cm,0.4493623405851461cm) rectangle (0.900225056264066cm,0.3743435858964741cm);
\filldraw[color=black!30, text=black] (0.900225056264066cm,2.099774943735934cm) rectangle (0.9752438109527382cm,2.024756189047262cm);
\filldraw[color=black!30, text=black] (0.900225056264066cm,2.024756189047262cm) rectangle (0.9752438109527382cm,1.9497374343585896cm);
\filldraw[color=black!30, text=black] (0.900225056264066cm,0.7494373593398351cm) rectangle (0.9752438109527382cm,0.6744186046511627cm);
\filldraw[color=black!30, text=black] (0.900225056264066cm,0.6744186046511627cm) rectangle (0.9752438109527382cm,0.5993998499624906cm);
\filldraw[color=black!30, text=black] (0.900225056264066cm,0.3743435858964741cm) rectangle (0.9752438109527382cm,0.29932483120780207cm);
\filldraw[color=black!30, text=black] (0.900225056264066cm,0.29932483120780207cm) rectangle (0.9752438109527382cm,0.2243060765191296cm);
\filldraw[color=black!30, text=black] (0.9752438109527382cm,2.099774943735934cm) rectangle (1.0502625656414104cm,2.024756189047262cm);
\filldraw[color=black!30, text=black] (0.9752438109527382cm,2.024756189047262cm) rectangle (1.0502625656414104cm,1.9497374343585896cm);
\filldraw[color=black!30, text=black] (0.9752438109527382cm,1.9497374343585896cm) rectangle (1.0502625656414104cm,1.8747186796699176cm);
\filldraw[color=black!30, text=black] (0.9752438109527382cm,0.6744186046511627cm) rectangle (1.0502625656414104cm,0.5993998499624906cm);
\filldraw[color=black!30, text=black] (0.9752438109527382cm,0.5993998499624906cm) rectangle (1.0502625656414104cm,0.5243810952738186cm);
\filldraw[color=black!30, text=black] (0.9752438109527382cm,0.29932483120780207cm) rectangle (1.0502625656414104cm,0.2243060765191296cm);
\filldraw[color=black!30, text=black] (0.9752438109527382cm,0.2243060765191296cm) rectangle (1.0502625656414104cm,0.14928732183045756cm);
\filldraw[color=black!30, text=black] (1.0502625656414104cm,2.024756189047262cm) rectangle (1.1252813203300824cm,1.9497374343585896cm);
\filldraw[color=black!30, text=black] (1.0502625656414104cm,1.9497374343585896cm) rectangle (1.1252813203300824cm,1.8747186796699176cm);
\filldraw[color=black!30, text=black] (1.0502625656414104cm,1.8747186796699176cm) rectangle (1.1252813203300824cm,1.7996999249812453cm);
\filldraw[color=black!30, text=black] (1.0502625656414104cm,0.5993998499624906cm) rectangle (1.1252813203300824cm,0.5243810952738186cm);
\filldraw[color=black!30, text=black] (1.0502625656414104cm,0.5243810952738186cm) rectangle (1.1252813203300824cm,0.4493623405851461cm);
\filldraw[color=black!30, text=black] (1.0502625656414104cm,0.2243060765191296cm) rectangle (1.1252813203300824cm,0.14928732183045756cm);
\filldraw[color=black!30, text=black] (1.0502625656414104cm,0.14928732183045756cm) rectangle (1.1252813203300824cm,0.07426856714178554cm);
\filldraw[color=black!30, text=black] (1.1252813203300824cm,1.9497374343585896cm) rectangle (1.2003000750187547cm,1.8747186796699176cm);
\filldraw[color=black!30, text=black] (1.1252813203300824cm,1.8747186796699176cm) rectangle (1.2003000750187547cm,1.7996999249812453cm);
\filldraw[color=black!30, text=black] (1.1252813203300824cm,0.5243810952738186cm) rectangle (1.2003000750187547cm,0.4493623405851461cm);
\filldraw[color=black!30, text=black] (1.1252813203300824cm,0.4493623405851461cm) rectangle (1.2003000750187547cm,0.3743435858964741cm);
\filldraw[color=black!30, text=black] (1.1252813203300824cm,0.14928732183045756cm) rectangle (1.2003000750187547cm,0.07426856714178554cm);
\filldraw[color=black!30, text=black] (1.1252813203300824cm,0.07426856714178554cm) rectangle (1.2003000750187547cm,-0.0007501875468864938cm);
\filldraw[color=black!30, text=black] (1.2003000750187547cm,1.7996999249812453cm) rectangle (1.275318829707427cm,1.724681170292573cm);
\filldraw[color=black!30, text=black] (1.2003000750187547cm,1.724681170292573cm) rectangle (1.275318829707427cm,1.649662415603901cm);
\filldraw[color=black!30, text=black] (1.2003000750187547cm,0.3743435858964741cm) rectangle (1.275318829707427cm,0.29932483120780207cm);
\filldraw[color=black!30, text=black] (1.2003000750187547cm,0.29932483120780207cm) rectangle (1.275318829707427cm,0.2243060765191296cm);
\filldraw[color=black!30, text=black] (1.275318829707427cm,1.7996999249812453cm) rectangle (1.350337584396099cm,1.724681170292573cm);
\filldraw[color=black!30, text=black] (1.275318829707427cm,1.724681170292573cm) rectangle (1.350337584396099cm,1.649662415603901cm);
\filldraw[color=black!30, text=black] (1.275318829707427cm,1.649662415603901cm) rectangle (1.350337584396099cm,1.5746436609152288cm);
\filldraw[color=black!30, text=black] (1.275318829707427cm,0.29932483120780207cm) rectangle (1.350337584396099cm,0.2243060765191296cm);
\filldraw[color=black!30, text=black] (1.275318829707427cm,0.2243060765191296cm) rectangle (1.350337584396099cm,0.14928732183045756cm);
\filldraw[color=black!30, text=black] (1.350337584396099cm,1.724681170292573cm) rectangle (1.4253563390847712cm,1.649662415603901cm);
\filldraw[color=black!30, text=black] (1.350337584396099cm,1.649662415603901cm) rectangle (1.4253563390847712cm,1.5746436609152288cm);
\filldraw[color=black!30, text=black] (1.350337584396099cm,1.5746436609152288cm) rectangle (1.4253563390847712cm,1.4996249062265568cm);
\filldraw[color=black!30, text=black] (1.350337584396099cm,0.2243060765191296cm) rectangle (1.4253563390847712cm,0.14928732183045756cm);
\filldraw[color=black!30, text=black] (1.350337584396099cm,0.14928732183045756cm) rectangle (1.4253563390847712cm,0.07426856714178554cm);
\filldraw[color=black!30, text=black] (1.4253563390847712cm,1.649662415603901cm) rectangle (1.5003750937734432cm,1.5746436609152288cm);
\filldraw[color=black!30, text=black] (1.4253563390847712cm,1.5746436609152288cm) rectangle (1.5003750937734432cm,1.4996249062265568cm);
\filldraw[color=black!30, text=black] (1.4253563390847712cm,0.14928732183045756cm) rectangle (1.5003750937734432cm,0.07426856714178554cm);
\filldraw[color=black!30, text=black] (1.4253563390847712cm,0.07426856714178554cm) rectangle (1.5003750937734432cm,-0.0007501875468864938cm);
\filldraw[color=black!30, text=black] (1.5003750937734432cm,3.0cm) rectangle (1.5753938484621155cm,2.924981245311328cm);
\filldraw[color=black!30, text=black] (1.5003750937734432cm,2.6999249812453114cm) rectangle (1.5753938484621155cm,2.6249062265566394cm);
\filldraw[color=black!30, text=black] (1.5003750937734432cm,1.4996249062265568cm) rectangle (1.5753938484621155cm,1.4246061515378845cm);
\filldraw[color=black!30, text=black] (1.5003750937734432cm,1.1245311327831957cm) rectangle (1.5753938484621155cm,1.0495123780945237cm);
\filldraw[color=black!30, text=black] (1.5753938484621155cm,3.0cm) rectangle (1.6504126031507877cm,2.924981245311328cm);
\filldraw[color=black!30, text=black] (1.5753938484621155cm,2.924981245311328cm) rectangle (1.6504126031507877cm,2.8499624906226555cm);
\filldraw[color=black!30, text=black] (1.5753938484621155cm,2.6999249812453114cm) rectangle (1.6504126031507877cm,2.6249062265566394cm);
\filldraw[color=black!30, text=black] (1.5753938484621155cm,2.6249062265566394cm) rectangle (1.6504126031507877cm,2.549887471867967cm);
\filldraw[color=black!30, text=black] (1.5753938484621155cm,1.4246061515378845cm) rectangle (1.6504126031507877cm,1.3495873968492123cm);
\filldraw[color=black!30, text=black] (1.5753938484621155cm,1.0495123780945237cm) rectangle (1.6504126031507877cm,0.9744936234058517cm);
\filldraw[color=black!30, text=black] (1.6504126031507877cm,2.924981245311328cm) rectangle (1.7254313578394598cm,2.8499624906226555cm);
\filldraw[color=black!30, text=black] (1.6504126031507877cm,2.8499624906226555cm) rectangle (1.7254313578394598cm,2.7749437359339835cm);
\filldraw[color=black!30, text=black] (1.6504126031507877cm,2.6249062265566394cm) rectangle (1.7254313578394598cm,2.549887471867967cm);
\filldraw[color=black!30, text=black] (1.6504126031507877cm,2.549887471867967cm) rectangle (1.7254313578394598cm,2.474868717179295cm);
\filldraw[color=black!30, text=black] (1.6504126031507877cm,1.3495873968492123cm) rectangle (1.7254313578394598cm,1.2745686421605402cm);
\filldraw[color=black!30, text=black] (1.6504126031507877cm,0.9744936234058517cm) rectangle (1.7254313578394598cm,0.8994748687171792cm);
\filldraw[color=black!30, text=black] (1.7254313578394598cm,2.8499624906226555cm) rectangle (1.800450112528132cm,2.7749437359339835cm);
\filldraw[color=black!30, text=black] (1.7254313578394598cm,2.7749437359339835cm) rectangle (1.800450112528132cm,2.6999249812453114cm);
\filldraw[color=black!30, text=black] (1.7254313578394598cm,2.549887471867967cm) rectangle (1.800450112528132cm,2.474868717179295cm);
\filldraw[color=black!30, text=black] (1.7254313578394598cm,2.474868717179295cm) rectangle (1.800450112528132cm,2.399849962490623cm);
\filldraw[color=black!30, text=black] (1.7254313578394598cm,1.2745686421605402cm) rectangle (1.800450112528132cm,1.199549887471868cm);
\filldraw[color=black!30, text=black] (1.7254313578394598cm,0.8994748687171792cm) rectangle (1.800450112528132cm,0.8244561140285072cm);
\filldraw[color=black!30, text=black] (1.800450112528132cm,2.7749437359339835cm) rectangle (1.8754688672168043cm,2.6999249812453114cm);
\filldraw[color=black!30, text=black] (1.800450112528132cm,2.474868717179295cm) rectangle (1.8754688672168043cm,2.399849962490623cm);
\filldraw[color=black!30, text=black] (1.800450112528132cm,1.199549887471868cm) rectangle (1.8754688672168043cm,1.1245311327831957cm);
\filldraw[color=black!30, text=black] (1.800450112528132cm,0.8244561140285072cm) rectangle (1.8754688672168043cm,0.7494373593398351cm);
\filldraw[color=black!30, text=black] (1.8754688672168043cm,2.6999249812453114cm) rectangle (1.9504876219054763cm,2.6249062265566394cm);
\filldraw[color=black!30, text=black] (1.8754688672168043cm,2.399849962490623cm) rectangle (1.9504876219054763cm,2.3248312078019504cm);
\filldraw[color=black!30, text=black] (1.8754688672168043cm,1.4996249062265568cm) rectangle (1.9504876219054763cm,1.4246061515378845cm);
\filldraw[color=black!30, text=black] (1.8754688672168043cm,1.1245311327831957cm) rectangle (1.9504876219054763cm,1.0495123780945237cm);
\filldraw[color=black!30, text=black] (1.8754688672168043cm,0.7494373593398351cm) rectangle (1.9504876219054763cm,0.6744186046511627cm);
\filldraw[color=black!30, text=black] (1.9504876219054763cm,2.6999249812453114cm) rectangle (2.0255063765941483cm,2.6249062265566394cm);
\filldraw[color=black!30, text=black] (1.9504876219054763cm,2.6249062265566394cm) rectangle (2.0255063765941483cm,2.549887471867967cm);
\filldraw[color=black!30, text=black] (1.9504876219054763cm,2.399849962490623cm) rectangle (2.0255063765941483cm,2.3248312078019504cm);
\filldraw[color=black!30, text=black] (1.9504876219054763cm,2.3248312078019504cm) rectangle (2.0255063765941483cm,2.2498124531132784cm);
\filldraw[color=black!30, text=black] (1.9504876219054763cm,1.4246061515378845cm) rectangle (2.0255063765941483cm,1.3495873968492123cm);
\filldraw[color=black!30, text=black] (1.9504876219054763cm,1.0495123780945237cm) rectangle (2.0255063765941483cm,0.9744936234058517cm);
\filldraw[color=black!30, text=black] (1.9504876219054763cm,0.6744186046511627cm) rectangle (2.0255063765941483cm,0.5993998499624906cm);
\filldraw[color=black!30, text=black] (2.0255063765941483cm,2.6249062265566394cm) rectangle (2.100525131282821cm,2.549887471867967cm);
\filldraw[color=black!30, text=black] (2.0255063765941483cm,2.549887471867967cm) rectangle (2.100525131282821cm,2.474868717179295cm);
\filldraw[color=black!30, text=black] (2.0255063765941483cm,2.3248312078019504cm) rectangle (2.100525131282821cm,2.2498124531132784cm);
\filldraw[color=black!30, text=black] (2.0255063765941483cm,2.2498124531132784cm) rectangle (2.100525131282821cm,2.1747936984246063cm);
\filldraw[color=black!30, text=black] (2.0255063765941483cm,1.3495873968492123cm) rectangle (2.100525131282821cm,1.2745686421605402cm);
\filldraw[color=black!30, text=black] (2.0255063765941483cm,0.9744936234058517cm) rectangle (2.100525131282821cm,0.8994748687171792cm);
\filldraw[color=black!30, text=black] (2.0255063765941483cm,0.5993998499624906cm) rectangle (2.100525131282821cm,0.5243810952738186cm);
\filldraw[color=black!30, text=black] (2.100525131282821cm,2.549887471867967cm) rectangle (2.175543885971493cm,2.474868717179295cm);
\filldraw[color=black!30, text=black] (2.100525131282821cm,2.474868717179295cm) rectangle (2.175543885971493cm,2.399849962490623cm);
\filldraw[color=black!30, text=black] (2.100525131282821cm,2.2498124531132784cm) rectangle (2.175543885971493cm,2.1747936984246063cm);
\filldraw[color=black!30, text=black] (2.100525131282821cm,2.1747936984246063cm) rectangle (2.175543885971493cm,2.099774943735934cm);
\filldraw[color=black!30, text=black] (2.100525131282821cm,1.2745686421605402cm) rectangle (2.175543885971493cm,1.199549887471868cm);
\filldraw[color=black!30, text=black] (2.100525131282821cm,0.8994748687171792cm) rectangle (2.175543885971493cm,0.8244561140285072cm);
\filldraw[color=black!30, text=black] (2.100525131282821cm,0.5243810952738186cm) rectangle (2.175543885971493cm,0.4493623405851461cm);
\filldraw[color=black!30, text=black] (2.175543885971493cm,2.474868717179295cm) rectangle (2.250562640660165cm,2.399849962490623cm);
\filldraw[color=black!30, text=black] (2.175543885971493cm,2.1747936984246063cm) rectangle (2.250562640660165cm,2.099774943735934cm);
\filldraw[color=black!30, text=black] (2.175543885971493cm,1.199549887471868cm) rectangle (2.250562640660165cm,1.1245311327831957cm);
\filldraw[color=black!30, text=black] (2.175543885971493cm,0.8244561140285072cm) rectangle (2.250562640660165cm,0.7494373593398351cm);
\filldraw[color=black!30, text=black] (2.175543885971493cm,0.4493623405851461cm) rectangle (2.250562640660165cm,0.3743435858964741cm);
\filldraw[color=black!30, text=black] (2.250562640660165cm,2.399849962490623cm) rectangle (2.3255813953488373cm,2.3248312078019504cm);
\filldraw[color=black!30, text=black] (2.250562640660165cm,2.099774943735934cm) rectangle (2.3255813953488373cm,2.024756189047262cm);
\filldraw[color=black!30, text=black] (2.250562640660165cm,1.1245311327831957cm) rectangle (2.3255813953488373cm,1.0495123780945237cm);
\filldraw[color=black!30, text=black] (2.250562640660165cm,0.7494373593398351cm) rectangle (2.3255813953488373cm,0.6744186046511627cm);
\filldraw[color=black!30, text=black] (2.250562640660165cm,0.3743435858964741cm) rectangle (2.3255813953488373cm,0.29932483120780207cm);
\filldraw[color=black!30, text=black] (2.3255813953488373cm,2.399849962490623cm) rectangle (2.4006001500375094cm,2.3248312078019504cm);
\filldraw[color=black!30, text=black] (2.3255813953488373cm,2.3248312078019504cm) rectangle (2.4006001500375094cm,2.2498124531132784cm);
\filldraw[color=black!30, text=black] (2.3255813953488373cm,2.099774943735934cm) rectangle (2.4006001500375094cm,2.024756189047262cm);
\filldraw[color=black!30, text=black] (2.3255813953488373cm,2.024756189047262cm) rectangle (2.4006001500375094cm,1.9497374343585896cm);
\filldraw[color=black!30, text=black] (2.3255813953488373cm,1.0495123780945237cm) rectangle (2.4006001500375094cm,0.9744936234058517cm);
\filldraw[color=black!30, text=black] (2.3255813953488373cm,0.6744186046511627cm) rectangle (2.4006001500375094cm,0.5993998499624906cm);
\filldraw[color=black!30, text=black] (2.3255813953488373cm,0.29932483120780207cm) rectangle (2.4006001500375094cm,0.2243060765191296cm);
\filldraw[color=black!30, text=black] (2.4006001500375094cm,2.3248312078019504cm) rectangle (2.4756189047261814cm,2.2498124531132784cm);
\filldraw[color=black!30, text=black] (2.4006001500375094cm,2.2498124531132784cm) rectangle (2.4756189047261814cm,2.1747936984246063cm);
\filldraw[color=black!30, text=black] (2.4006001500375094cm,2.024756189047262cm) rectangle (2.4756189047261814cm,1.9497374343585896cm);
\filldraw[color=black!30, text=black] (2.4006001500375094cm,1.9497374343585896cm) rectangle (2.4756189047261814cm,1.8747186796699176cm);
\filldraw[color=black!30, text=black] (2.4006001500375094cm,0.9744936234058517cm) rectangle (2.4756189047261814cm,0.8994748687171792cm);
\filldraw[color=black!30, text=black] (2.4006001500375094cm,0.5993998499624906cm) rectangle (2.4756189047261814cm,0.5243810952738186cm);
\filldraw[color=black!30, text=black] (2.4006001500375094cm,0.2243060765191296cm) rectangle (2.4756189047261814cm,0.14928732183045756cm);
\filldraw[color=black!30, text=black] (2.4756189047261814cm,2.2498124531132784cm) rectangle (2.550637659414854cm,2.1747936984246063cm);
\filldraw[color=black!30, text=black] (2.4756189047261814cm,2.1747936984246063cm) rectangle (2.550637659414854cm,2.099774943735934cm);
\filldraw[color=black!30, text=black] (2.4756189047261814cm,1.9497374343585896cm) rectangle (2.550637659414854cm,1.8747186796699176cm);
\filldraw[color=black!30, text=black] (2.4756189047261814cm,1.8747186796699176cm) rectangle (2.550637659414854cm,1.7996999249812453cm);
\filldraw[color=black!30, text=black] (2.4756189047261814cm,0.8994748687171792cm) rectangle (2.550637659414854cm,0.8244561140285072cm);
\filldraw[color=black!30, text=black] (2.4756189047261814cm,0.5243810952738186cm) rectangle (2.550637659414854cm,0.4493623405851461cm);
\filldraw[color=black!30, text=black] (2.4756189047261814cm,0.14928732183045756cm) rectangle (2.550637659414854cm,0.07426856714178554cm);
\filldraw[color=black!30, text=black] (2.550637659414854cm,2.1747936984246063cm) rectangle (2.625656414103526cm,2.099774943735934cm);
\filldraw[color=black!30, text=black] (2.550637659414854cm,1.8747186796699176cm) rectangle (2.625656414103526cm,1.7996999249812453cm);
\filldraw[color=black!30, text=black] (2.550637659414854cm,0.8244561140285072cm) rectangle (2.625656414103526cm,0.7494373593398351cm);
\filldraw[color=black!30, text=black] (2.550637659414854cm,0.4493623405851461cm) rectangle (2.625656414103526cm,0.3743435858964741cm);
\filldraw[color=black!30, text=black] (2.550637659414854cm,0.07426856714178554cm) rectangle (2.625656414103526cm,-0.0007501875468864938cm);
\filldraw[color=black!30, text=black] (2.625656414103526cm,2.099774943735934cm) rectangle (2.700675168792198cm,2.024756189047262cm);
\filldraw[color=black!30, text=black] (2.625656414103526cm,1.7996999249812453cm) rectangle (2.700675168792198cm,1.724681170292573cm);
\filldraw[color=black!30, text=black] (2.625656414103526cm,0.7494373593398351cm) rectangle (2.700675168792198cm,0.6744186046511627cm);
\filldraw[color=black!30, text=black] (2.625656414103526cm,0.3743435858964741cm) rectangle (2.700675168792198cm,0.29932483120780207cm);
\filldraw[color=black!30, text=black] (2.700675168792198cm,2.099774943735934cm) rectangle (2.7756939234808704cm,2.024756189047262cm);
\filldraw[color=black!30, text=black] (2.700675168792198cm,2.024756189047262cm) rectangle (2.7756939234808704cm,1.9497374343585896cm);
\filldraw[color=black!30, text=black] (2.700675168792198cm,1.7996999249812453cm) rectangle (2.7756939234808704cm,1.724681170292573cm);
\filldraw[color=black!30, text=black] (2.700675168792198cm,1.724681170292573cm) rectangle (2.7756939234808704cm,1.649662415603901cm);
\filldraw[color=black!30, text=black] (2.700675168792198cm,0.6744186046511627cm) rectangle (2.7756939234808704cm,0.5993998499624906cm);
\filldraw[color=black!30, text=black] (2.700675168792198cm,0.29932483120780207cm) rectangle (2.7756939234808704cm,0.2243060765191296cm);
\filldraw[color=black!30, text=black] (2.7756939234808704cm,2.024756189047262cm) rectangle (2.8507126781695424cm,1.9497374343585896cm);
\filldraw[color=black!30, text=black] (2.7756939234808704cm,1.9497374343585896cm) rectangle (2.8507126781695424cm,1.8747186796699176cm);
\filldraw[color=black!30, text=black] (2.7756939234808704cm,1.724681170292573cm) rectangle (2.8507126781695424cm,1.649662415603901cm);
\filldraw[color=black!30, text=black] (2.7756939234808704cm,1.649662415603901cm) rectangle (2.8507126781695424cm,1.5746436609152288cm);
\filldraw[color=black!30, text=black] (2.7756939234808704cm,0.5993998499624906cm) rectangle (2.8507126781695424cm,0.5243810952738186cm);
\filldraw[color=black!30, text=black] (2.7756939234808704cm,0.2243060765191296cm) rectangle (2.8507126781695424cm,0.14928732183045756cm);
\filldraw[color=black!30, text=black] (2.8507126781695424cm,1.9497374343585896cm) rectangle (2.9257314328582145cm,1.8747186796699176cm);
\filldraw[color=black!30, text=black] (2.8507126781695424cm,1.8747186796699176cm) rectangle (2.9257314328582145cm,1.7996999249812453cm);
\filldraw[color=black!30, text=black] (2.8507126781695424cm,1.649662415603901cm) rectangle (2.9257314328582145cm,1.5746436609152288cm);
\filldraw[color=black!30, text=black] (2.8507126781695424cm,1.5746436609152288cm) rectangle (2.9257314328582145cm,1.4996249062265568cm);
\filldraw[color=black!30, text=black] (2.8507126781695424cm,0.5243810952738186cm) rectangle (2.9257314328582145cm,0.4493623405851461cm);
\filldraw[color=black!30, text=black] (2.8507126781695424cm,0.14928732183045756cm) rectangle (2.9257314328582145cm,0.07426856714178554cm);
\filldraw[color=black!30, text=black] (2.9257314328582145cm,1.8747186796699176cm) rectangle (3.0007501875468865cm,1.7996999249812453cm);
\filldraw[color=black!30, text=black] (2.9257314328582145cm,1.5746436609152288cm) rectangle (3.0007501875468865cm,1.4996249062265568cm);
\filldraw[color=black!30, text=black] (2.9257314328582145cm,0.4493623405851461cm) rectangle (3.0007501875468865cm,0.3743435858964741cm);
\filldraw[color=black!30, text=black] (2.9257314328582145cm,0.07426856714178554cm) rectangle (3.0007501875468865cm,-0.0007501875468864938cm);
\node[draw=none,align=center, anchor=north] at (1.5,-0.1) {\footnotesize $\symbf{\lambda}_\textbf{A} \in \mathbb{R}^{6\,720 \times 6\,720}$ \\ \footnotesize $\bar{n} = 840$, $\ell^2 = 25$, \\ \footnotesize $\text{nnz} = 7.5\mathrm{e}6$};
\end{tikzpicture}



	\caption{The growth of $\symbf{\lambda}_\textbf{A}$. For a constant number of grid points the number of non-zero entries in this matrix increases when increasing the number of elements.}
	\label{fig:lambda_diagram}
\end{figure}

%
%
%
\subsubsection{Computing the trace inverse product}

%
%
%
\noindent
In our implementation, the product $\textbf{M}^{-1}\textbf{F}$ in \eqnref{eqn:global_system_a} is computed from a series a series of solves, storing $\textbf{F}$ as a vector of vectors to compute $\textbf{M}^{-1}\textbf{F}$. 
That is, for each directional $\textbf{F}^k$ boundary we store a list of $n$ vectors
\begin{equation}
	\textbf{F}^k = \left[\textbf{f}^k_1 \cdots \textbf{f}^k_n\right] 
	\in \mathbb{R}^{n \times n^2}.
\end{equation} 
\noindent
Here, each vector $\textbf{f}^k_i$ is implemented as a PetscVector object.
This is in contrast to storing $\textbf{F}$ as a sparse matrix, and is necessary for the computation of $\textbf{M}^{-1}\textbf{F}$ from the local 
matrices of $\textbf{M}$, stored similarly as
\begin{equation}
	\textbf{M} = \left\{\textbf{M}_1 \cdots \textbf{M}_p\right\} \in 
	\mathbb{R}^{\ell^2 n^2 \times \ell^2 n^2}, p \leq \ell^2.
\end{equation} 
\noindent 
Where $p$ denotes the number of unique local matrix operators in 
$\textbf{M}$. 

%
%
%
With this we compute the intermediate result $\textbf{M}^{-1}\textbf{F}$, performing a linear solve of every unique local matrix operator, and every vector of every directional boundary coefficient matrix 
\begin{equation}
	\text{solve}(\textbf{M}_{i}, \textbf{F}^{k}_j) \text{  for }
	\begin{array}{l}
		0 < i \leq p, \\
		0 < j \leq n, \\
		0 < k \leq 4. \\ 
	\end{array}
	\label{eqn:mfsolves}
\end{equation}
\noindent
This result has a similar non-zero pattern as $\textbf{F}$, following the same symbolic block structure. 
In our implementation we solve this system through direct solvers made available through PETSc. 

%
%
%
\begin{aside}
	In the homogeneous Poisson equation in \eqnref{eqn:problem_desc} we have 
	\begin{equation}
		p = \begin{cases}
		    \ell &  0 \leq \ell < 3 \\ 
			3 &  3 \leq \ell \\ 
		\end{cases}.
	\end{equation} 
	\noindent 
	When $p = 3$ we have one unique $\textbf{M}_i$ for the elements along the top Neumann boundaries, the bottom Neumann boundaries, and the interior elements. 
	Since $p$ is constant when $\ell \geq 3$, increasing the number of elements, $\ell^2$ decreases the number of solves in \eqnref{eqn:mfsolves}, as well as the size of each solve, for a constant global volume, $\bar{n}^2$. 
\end{aside}

%
%
%
\noindent
The number of solves we compute for $\textbf{M}^{-1}\textbf{F}^\intercal$ is then $p \times 3 \times n$.
Thus the total FLOPS we use to compute elements of size $n^2 \times n^2$ using forwards/backwards substitution is 
\begin{equation}
	\text{FLOPS}(\textbf{M}^{-1}\textbf{F}^\intercal) = 24 (\bar{n}/\ell)^5.
	\label{eqn:flops_mf}
\end{equation}
\noindent
Since we fix the problem size $\bar{n}^2$, we write $n = \bar{n}/\ell$, allowing us to parameterize our model by the number of elements instead of the size of each element. 
As $\bar{n}$ is constant, computing $\textbf{M}^{-1}\textbf{F}^\intercal$ will require proportionally fewer FLOPS as the number of elements increases.

%
%
%
\subsubsection{Computing the trace MatMul} 

%
%
%
The pattern of the $\textbf{F}$ matrix is determined by the interfaces since we compute the product of each intermediary result in $\textbf{M}^{-1} \textbf{F}$ with the $\textbf{F}^k$ interfaces that make a face in a given $\textbf{M}_i$. 
This is illustrated by \figref{fig:block_diagram} where we see that each shaded row in $\textbf{F}$ corresponds to an interface in \figref{fig:volume_diagram}.
Again, we write the number of interfaces in terms of the number of elements, $r = 2\ell^2 - 2\ell$. 
The dimensions of $\symbf{\lambda}_\textbf{A}$ can then be written as 
\begin{equation}
	\symbf{\lambda}_\textbf{A} \in \mathbb{R}^{rn \times rn} \equiv \mathbb{R}^{2\bar{n}\ell - 2\bar{n} \times 2\bar{n}\ell - 2\bar{n}}. 
	\label{eqn:lamsize}
\end{equation}
%The non-zero structure of the matrix used to solve $\symbf{\lambda}$ in \eqnref{eqn:global_system_a} provides insight into the complexity the two operations in $\textbf{F}^{\intercal} \times \textbf{M}^{-1} \textbf{F}$. 
%The memory needed to store the decomposed matrix consisting of $\textbf{M}$, $\textbf{F}$, $\textbf{F}^{\intercal}$, $\textbf{D}$ is minimally the size of each unique, decomposed sub-matrix. 
%This is illustrated in \figref{fig:block_diagram}, as the $\textbf{F}$ and $\textbf{F}^{\intercal}$ matrices only contain 4 unique sub-matrices, $\textbf{F}^{1}$, $\textbf{F}^{2}$, $\textbf{F}^{3}$, and $\textbf{F}^{4}$. 
%The uniqueness of $\textbf{M}$ is less obvious here, but by \figref{fig:volume_diagram} we see that there are only 3 unique local problems, one with a Neumann boundary on face 4, another with a Neumann boundary on face 3, and a final problem with a no Neumann boundary conditions.
%Fixing the number of global volume points, $\bar{n}^2$, the number of local volume points per row, $n$, is given in terms of the number of elements per row, $n = \bar{n} / \ell$, and similarly 
\noindent
The number of non-zeros in the intermediary result $\textbf{F}^{\intercal}(\textbf{M}^{-1}\textbf{F})$ is given by
\begin{subequations}
\begin{equation}
	n^2 \sum_{i=1}^{\ell^2} (\phi_i^2)
	\label{eqn:nz_sum}
\end{equation}
where
\begin{equation}
	\textbf{F}^{\text{ind}} = \left[\textbf{f}^{\text{ind}}_1 \cdots \textbf{f}^{\text{ind}}_{\ell^2}\right] \in \mathbb{Z}^{\ell^2 \times r},
\end{equation}
\begin{equation}
	\phi = \left[ \text{nnz}(\textbf{f}^{\text{ind}}_1) \cdots \text{nnz}(\textbf{f}^{\text{ind}}_{\ell^2}) \right] \in \mathbb{Z}^{\ell^2}, 
\end{equation}
\end{subequations}
or, the sum of the square of the number of internal interfaces per element. 
The additional constraints we place on this problem, that being the square orientation of elements, we can write $\phi$ as 
\begin{equation}
	\phi = \left[ \underbrace{2, 2, 2, 2}_{4}, \underbrace{3 \cdots 3}_{4\ell - 8}, \underbrace{4 \cdots 4}_{(\ell-2)^2} \right] \in \mathbb{Z}^{\ell^2}, 
\end{equation}
where $2\text{'s}$ denote the corner elements, $3\text{'s}$ denote the boundary elements, and $4\text{'s}$ denote the internal elements. 
In \eqnref{eqn:nz_sum} we square each $\phi_i$ because $\textbf{F}^{\intercal}$ and  $\textbf{M}^{-1}\textbf{F}$ can both be derived from substitutions of $\textbf{F}^{\text{ind}}$, and the sum is multiplied by $n^2$ accounting for the additional dimension of each $\textbf{F}^k$. 

%
%  NOTE: PUTTING IT HERE TO LOOK NICE IT'S NOT LOGICALLY RELATED
%
\begin{figure*}
\centering
\begin{tabular}{lc}
& \small Total elements \\
\rotatebox[origin=c]{90}{\small Threads \kern 1em} & {\pgfplotstabletypeset[
    font=\footnotesize,
    color cells={
        min = 1.44, 
        max = 2.5},
    /pgfplots/colormap = {whitered}{
        rgb255(0cm) = (120, 40, 40); rgb255(1cm) = (200, 200, 200)
    },
    /color cells/textcolor/.initial = {white},
    every head row/.style={
    typeset cell/.code={%% add the word 'Rank'...
    \ifnum\pgfplotstablecol=\pgfplotstablecols%
    \pgfkeyssetvalue{/pgfplots/table/@cell content}{1600 \\}%
    \else%
    \ifnum\pgfplotstablecol=1%
    \pgfkeyssetvalue{/pgfplots/table/@cell content}{&}%
    \else%
    \ifnum\pgfplotstablecol=2%
    \pgfkeyssetvalue{/pgfplots/table/@cell content}{ 9 &}%
    \else%
    \ifnum\pgfplotstablecol=3%
    \pgfkeyssetvalue{/pgfplots/table/@cell content}{ 16 &}%
    \else%
    \ifnum\pgfplotstablecol=4%
    \pgfkeyssetvalue{/pgfplots/table/@cell content}{ 25 &}%
    \else%
    \ifnum\pgfplotstablecol=5%
    \pgfkeyssetvalue{/pgfplots/table/@cell content}{ 36 &}%
    \else%
    \ifnum\pgfplotstablecol=6%
    \pgfkeyssetvalue{/pgfplots/table/@cell content}{ 49 &}%
    \else%
    \ifnum\pgfplotstablecol=7%
    \pgfkeyssetvalue{/pgfplots/table/@cell content}{ 64 &}%
    \else%
    \ifnum\pgfplotstablecol=8%
    \pgfkeyssetvalue{/pgfplots/table/@cell content}{ 100 &}%
    \else%
    \ifnum\pgfplotstablecol=9%
    \pgfkeyssetvalue{/pgfplots/table/@cell content}{ 144 &}%
    \else%
    \ifnum\pgfplotstablecol=10%
    \pgfkeyssetvalue{/pgfplots/table/@cell content}{ 196 &}%
    \else%
    \ifnum\pgfplotstablecol=11%
    \pgfkeyssetvalue{/pgfplots/table/@cell content}{ 225 &}%
    \else%
    \ifnum\pgfplotstablecol=12%
    \pgfkeyssetvalue{/pgfplots/table/@cell content}{ 400 &}%
    \else%
    \ifnum\pgfplotstablecol=13%
    \pgfkeyssetvalue{/pgfplots/table/@cell content}{ 441 &}%
    \else%
    \ifnum\pgfplotstablecol=14%
    \pgfkeyssetvalue{/pgfplots/table/@cell content}{ 576 &}%
    \else%
    \ifnum\pgfplotstablecol=15%
    \pgfkeyssetvalue{/pgfplots/table/@cell content}{ 784 &}%
    \else%
    \ifnum\pgfplotstablecol=16%
    \pgfkeyssetvalue{/pgfplots/table/@cell content}{ 900 &}%
    \else%
    \ifnum\pgfplotstablecol=17%
    \pgfkeyssetvalue{/pgfplots/table/@cell content}{ 1225 &}%
    \else%
    \pgfkeyssetvalue{/pgfplots/table/@cell content}{1600 \\}%
    \fi\fi\fi\fi\fi\fi\fi\fi\fi\fi\fi\fi\fi\fi\fi\fi\fi\fi%
    }%%
    },
    col sep = comma,
    columns/0/.style={reset styles}
]{./data/tts}} \\
\end{tabular}
\caption{Time to solution (seconds) given fixed problem size for various 
    element and thread configurations on a single Xeon E5-2690 node.}
\end{figure*}


%,9,16,25,36,49,64,100,144,196,225,400,441,576,784,900,1225,1600


%
%
%
The ordering of the rows in $\textbf{F}^{\text{ind}}$ may not be equivalent in this case, but the final sum will be identical. 
Throughout the rest of the problem this is not an issue if the ordering remains consistent. 

%
%
%
From the number of non-zeros in $\symbf{\lambda}_\textbf{A}$ we derive the total FLOPS to compute $\textbf{F}^{\intercal} \times (\textbf{M}^{-1}\textbf{F})$ as a $v^\intercal v$ product of every $n^2$ length row in each $(\textbf{F}^k)^{\intercal}$ and each $n^2$ length column in the intermediary matrices of $(\textbf{M}^{-1}\textbf{F})$.
This is again, expressed as a function of the number of elements
\begin{equation}
	\text{FLOPS}(\textbf{F}^{\intercal} \times (\textbf{M}^{-1}\textbf{F})) = 2 (\bar{n}/\ell)^4 \sum_{i=1}^{\ell^2} (\phi_i^2).
	\label{eqn:flops_fmf}
\end{equation}

%
%
%
\begin{figure}
	{\centering 
	\begin{center}
\begin{tikzpicture}
\begin{axis}[
  /pgf/number format/1000 sep={\,},
  height=2in,
    xlabel={\small $\ell^2$ (total elements)},
  width =3in,
  axis x line*=bottom,
  axis y line*=left,
  xtick={9, 144, 484, 1024},
  %ymode=log,
  ylabel={\small $\text{FLOP} / \text{byte}$}
  legend style={
    font=\footnotesize,
    at={(0.975, 0.05)},
    anchor=south east},
  legend cell align={left}]
    \addplot[
        select coords between index={1}{2500},
        color=black,smooth,thick] 
        table[col sep=comma,header=false,
        x index=0,y index=1] {data/ai_fmft};
\end{axis}
\end{tikzpicture}
\end{center}
}
	\caption{\textbf{The arithmetic intensity ($\text{FLOPs}/\text{byte}$) of $\textbf{F}^\intercal \times (\textbf{M}^{-1} \textbf{F})$.}}
	\label{fig:ftmt_ai}
\end{figure}

\noindent
Here, we can also discretely determine the minimum number of bytes needed by this operation. 
That is, the memory needed for $\textbf{F}^{\intercal}$, $(\textbf{M}^{-1}\textbf{F})$, and $\symbf{\lambda}_\textbf{A}$

\begin{equation}
	\text{BYTES}(\textbf{F}^{\intercal} \times (\textbf{M}^{-1}\textbf{F})) = 8 \times 16 (\bar{n}/\ell)^3 + 8 \times (\bar{n}/\ell)^2 \sum_{i=1}^{\ell^2} (\phi_i^2).
	\label{eqn:bytes_fmf}
\end{equation}

\noindent 
From this we derive the arithmetic intensity of this operation  

\begin{equation}
	\symbf{\beta}(\textbf{F}^{\intercal} \times (\textbf{M}^{-1}\textbf{F})) = \frac{\text{FLOPS}(\textbf{F}^{\intercal} \times (\textbf{M}^{-1}\textbf{F}))}{\text{BYTES}(\textbf{F}^{\intercal} \times (\textbf{M}^{-1}\textbf{F}))}
\end{equation}

\noindent
which we plot as a function of the number of elements in \figref{fig:ftmt_ai}.


%With this, we have the non-zero density of $\textbf{F}^{\intercal}\textbf{M}^{-1}\textbf{F}$ as a function of the number of elements per row 
%\begin{equation}
%	\begin{aligned}
%	\text{nzd}(\ell) &= \frac{\bar{n}/\ell (16\ell^2 - 28\ell + 8)}{(2\bar{n}\ell - 2\bar{n})^2} \\
%	&= \frac{4\ell^2 - 7\ell + 2}{\bar{n}\ell(\ell-1)^2}.
%	\end{aligned}
%\end{equation}

%\begin{figure}
%	{\centering
%	%\begin{subfigure}{\columnwidth}
%	%	\centering
%	%	\pgfplotsset{compat=newest, width=\columnwidth}
\begin{tikzpicture}
\begin{axis}[
  /pgf/number format/1000 sep={\,},
  height=2in,
	xlabel={\small $\ell^2$ (total elements)},
  xtick={9, 144, 484, 1024},
  ymode=log,
  ylabel={\small Non-zero density of $\symbf{\lambda}_{\textbf{A}} \text{(log)}$},
  legend style={
    font=\footnotesize}, 
  legend cell align={left}]
    \addplot[
    	select coords between index={1}{2500},
    	color=red,smooth,thick] 
    	table[col sep=comma,header=false,
      	x index=0,y index=1] {data/density1};
    \addlegendentry{$\bar{n}^2 = 1\text{E+}4$}
    \addplot[
    	select coords between index={1}{2500},
    	color=blue,smooth,thick] table[col sep=comma,header=false,
      x index=0,y index=1] {data/density2};
    \addlegendentry{$\bar{n}^2 = 1\text{E+}5$}
    \addplot[
    	select coords between index={1}{2500},
    	color=black,smooth,thick] table[col sep=comma,header=false,
      	x index=0,y index=1] {data/density3};
    \addlegendentry{$\bar{n}^2 = 1\text{E+}6$}
\end{axis}
\end{tikzpicture}
%	%\end{subfigure}
%	\begin{subfigure}{\columnwidth}
%		\centering
%		\pgfplotsset{compat=newest, width=\columnwidth}
\begin{tikzpicture}
\begin{axis}[height=2in,
	xlabel={$\ell^2$ (total elements)},
  xtick={9, 144, 484, 1024},
    ylabel={Non-zeros in $\symbf{\lambda}_{\textbf{A}}$},
    legend style={font=\footnotesize},
    legend cell align={left}]
    \addplot[
    	select coords between index={1}{2500},
    	color=red,smooth,thick] 
    	table[col sep=comma,header=false,
      	x index=0,y index=1] {data/nonzeros1};
    \addlegendentry{$\bar{n}^2 = 10000$}
    \addplot[
    	select coords between index={1}{2500},
    	color=blue,smooth,thick] table[col sep=comma,header=false,
      x index=0,y index=1] {data/nonzeros2};
    \addlegendentry{$\bar{n}^2 = 100000$}
    \addplot[
    	select coords between index={1}{2500},
    	color=black,smooth,thick] table[col sep=comma,header=false,
      	x index=0,y index=1] {data/nonzeros3};
    \addlegendentry{$\bar{n}^2 = 1000000$}
\end{axis}
\end{tikzpicture}
%	\end{subfigure}}
%	\caption[The non-zero density of $\symbf{\lambda}_{\textbf{A}}$ 
%	plotted as a function of the total number of elements.]{\textbf{The non-zeros 
%	elements of $\symbf{\lambda}_{\textbf{A}}$ plotted as a function of 
%	the total number of elements.}}
%	\label{fig:nz_density} 
%\end{figure}

%               TODO: mention preconditioning
%\noindent
%This is plotted in \figref{fig:nz_density} for several values of $\bar{n}$ alongside of a plot of the quantity of non-zeros. Here we see that when the $\bar{n}^2$ is fixed, the size of the trace system is governed by a 3rd degree polynomial while the quantity of non-zeros is 2nd order, resulting in different non-zero densities depending on the $\ell$. 
%This is significant as performance often depends on the choice of solvers and preconditioners and that are well suited to the size of non-zero pattern of the problem \citep{bollhofer2020state}. % todo: find more sources here 

%In our implementation each non-zero corresponds to a vector dot product between a non-zero row of the sub-matrices of $\textbf{F}^{\intercal}$ and a decomposed result of computing  $\textbf{M}^{-1}\textbf{F}$

%
%
%
\subsubsection{Computing the trace vector}

%
%
%
We utilize two potentially expensive kernels to compute $\symbf{\lambda}_{\textbf{b}}$ in \eqnref{eqn:global_system_c}, those being a matrix-vector linear solve, similar to the matrix-matrix solve in \eqnref{eqn:global_system_b}, and an SpMV. 
The complexity of this is a solve is three polynomial orders lower than \eqnref{eqn:flops_mf} at
\begin{equation}
	\text{FLOPS}(\textbf{M}^{-1}\bar{\textbf{g}}) = \bar{n}^4/\ell^2.
\end{equation}
\noindent
This is evident as $\bar{\textbf{g}}$ is a length $\bar{n}^2$ vector, not a matrix like $\textbf{F}$; as $\bar{\textbf{g}}$ is a unique vector for every $\textbf{M}_i$ we compute $\ell^2$ solves instead of $p \times 3 \times n$ solves.

%
%
%
To solve the SpMV we once again utilize the block diagonal structure of the problem to compute $\textbf{F}^\intercal \times (\textbf{M}^{-1}\bar{\textbf{g}})$. 
We compute each component solution from each $\textbf{F}^k$ to the appropriate intermediate vectors of $\textbf{M}^{-1}\bar{\textbf{g}}$.
FLOPs are totaled through a sum similar to \eqnref{eqn:nz_sum}, but only multiplied by $n$, as this is a matrix-vector operation.  
The total FLOPs to compute this operation is then 
\begin{equation}
	\text{FLOPS}(\textbf{F}^\intercal \times (\textbf{M}^{-1}\bar{\textbf{g}})) = 12(\bar{n}/\ell) \sum_{i=1}^{\ell^2} (\phi_i^2).
\end{equation}
This appears as a 1st order polynomial because each $\textbf{F}^k$ contains $6(\bar{n}/\ell)$ entries is used only once for each entry in $\textbf{F}^{\text{ind}}$.
We also then have the number of bytes needed by SpMV in this instance
\begin{equation}
	\text{BYTES}(\textbf{F}^\intercal \times (\textbf{M}^{-1}\bar{\textbf{g}})) = 96(\bar{n}/\ell) \sum_{i=1}^{\ell^2} (\phi_i^2).
\end{equation}

%
%
%
\subsubsection{Computing the trace solve}

%
%
%
To solve $\symbf{\lambda}_{\textbf{A}}$ we use a direct Cholesky solver provided through PETSc. 
Here the total FLOPs to solve a $2\bar{n}\ell - 2\bar{n} \times 2\bar{n}\ell - 2\bar{n}$ system is 

\begin{equation}
	\text{FLOPS}(\symbf{\lambda}_{\textbf{A}}^{-1}\symbf{\lambda}_{\textbf{b}}) = 2  (2\bar{n}\ell - 2\bar{n})^2.
\end{equation}

%
%
%
\subsubsection{Computing the local solves}

%
%
%
Finally, to compute the solution via the local systems in PETSc's Cholesky solver in \eqnref{eqn:local_system} we have $\ell$, $n^2 \times n^2$ component matrices of $\textbf{M}$, by $\ell$ sub-sections of the vector $(\bar{\textbf{g}} - \textbf{F} \symbf{\lambda})$. 
The total FLOPs are then 
\begin{equation}
	\text{FLOPS}(\textbf{M}^{-1}(\bar{\textbf{g}} - \textbf{F} \symbf{\lambda})) = 2 (\bar{n}^4/\ell^5).
\end{equation}

%\noindent
%From \eqnref{eqn:nz_sum} we have the total number of non-zeros in the intermediary matrix $\textbf{F}^\intercal \textbf{M}^{-1} \textbf{F}$ which is closely representative of $\symbf{\lambda}_{\textbf{A}}$, only lacking entries from $\textbf{D}$.  
%$\textbf{D}$ however is single diagonal and all entries of $\textbf{D}$ are already non-zero in $\textbf{F}^\intercal \textbf{M}^{-1} \textbf{F}$.
%
%\begin{equation}
%	\text{MEMOPS}(\symbf{\lambda}_{\textbf{A}}^{-1}\symbf{\lambda}_{\textbf{b}}s) = n + n^2 \sum_{i=1}^{\ell^2} (\phi_i^2)
%\end{equation}






