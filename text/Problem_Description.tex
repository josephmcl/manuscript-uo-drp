%
%
%
Through the use of the hybridized SBP-SAT method, we investigate the shared memory, strong-scaling characteristics of the 2D Poisson equation assembled with the SBP-SAT method \citep{kozdon2021hybridized}. 
Whereas PDE performance is largely considered in a weak-scaling context, we study the strong-scaling characteristics by fixing the number of volume points, leveraging the hybrid formulation of the SBP-SAT method, to solve a varied size and quantity of independent problems.

%
%
%
\noindent 
We assemble a 2D Poisson equation 
\begin{subequations}
\begin{equation}
	\left( \dfrac{\partial^2}{\partial x^2} + \dfrac{\partial^2}{\partial y^2} \right) u(x, y) = f(x, y) 
\end{equation}
\noindent 
defined over the domain
\begin{equation}
	0 \leq x \leq 1,
\end{equation}
\begin{equation}
	0 \leq y \leq 1,
\end{equation}
\noindent
given the boundary conditions
\begin{equation}
	u(0, y) = \text{sin}(y),
\end{equation}
\begin{equation}
	u(1, y) = \text{sin}(\pi + y),
\end{equation}
\begin{equation}
	\dfrac{\partial u(x, 0)}{\partial y} = -\pi \text{cos}(\pi x),
\end{equation}
\begin{equation}
	\dfrac{\partial u(x, 1)}{\partial y} = -\pi \text{cos}(\pi x)
\end{equation}
\noindent 
and source function
\begin{equation}
	f(x, y) = -2 \pi^2 u(x, y). 
\end{equation}
	\label{eqn:problem_desc}
\end{subequations}
\noindent 
This sets up a problem utilizing the manufactured solution
\begin{equation}
	u(x, y) = sin(\pi x + \pi y).
\end{equation}