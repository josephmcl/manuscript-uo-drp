\section{Method}

Through the use of the hybrid method derived by [cite], we investigate the shared memory, strong-scaling characteristics of the 2D Poisson equation assembled with the SBP-SAT method. Whereas PDE performance is increasingly studied in a weak-scaling context, we leverage the hybrid formulation of SBP-SAT, decoupling each element to be solved independently. This invokes three research questions:

\begin{enumerate}[label=\textbf{RQ \arabic*.}]
	\item {Do we observe strong-scaling with the hybrid \mbox{SBP-SAT} method by varying element size?}
	\item {Does the ideal element size vary w.r.t. the problem size?}
	\item {Is the ideal element size hardware dependent?}
\end{enumerate}

\subsection{Problem description}


\noindent 
We assemble a 2D Poisson equation 
\begin{subequations}
\begin{equation}
	\left( \dfrac{\partial^2}{\partial x^2} + \dfrac{\partial^2}{\partial y^2} \right) u(x, y) = f(x, y) 
\end{equation}
\noindent 
defined over the domain
\begin{equation}
	0 \leq x \leq 1,
\end{equation}
\begin{equation}
	0 \leq y \leq 1,
\end{equation}
\noindent
given the boundary conditions
\begin{equation}
	u(0, y) = \text{sin}(y),
\end{equation}
\begin{equation}
	u(1, y) = \text{sin}(\pi + y),
\end{equation}
\begin{equation}
	\dfrac{\partial u(x, 0)}{\partial y} = -\pi \text{cos}(\pi x),
\end{equation}
\begin{equation}
	\dfrac{\partial u(x, 1)}{\partial y} = -\pi \text{cos}(\pi x)
\end{equation}
\noindent 
and source function
\begin{equation}
	f(x, y) = -2 \pi^2 u(x, y). 
\end{equation}
\end{subequations}
\noindent 
This sets up the manufactured solution
\begin{equation}
	u(x, y) = sin(\pi x + \pi y).
\end{equation}

%%%%%%%%%%%%%%%%%%%%%%%%%%%%%%%%%%%%%%%%%%%%%%%%%%%%%%%%%%%%%%%%%%%%%
%%%%%%%%%%%%%%%%%%%%%%%%%%%%%%%%%%%%%%%%%%%%%%%%%%%%%%%%%%%%%%%%%%%%%
\subsection{Hybrid scheme}
%%%%%%%%%%%%%%%%%%%%%%%%%%%%%%%%%%%%%%%%%%%%%%%%%%%%%%%%%%%%%%%%%%%%%
%%%%%%%%%%%%%%%%%%%%%%%%%%%%%%%%%%%%%%%%%%%%%%%%%%%%%%%%%%%%%%%%%%%%%

The hybridized SBP-SAT scheme discretizes this problem into the form 
\begin{equation}
    \left[\begin{array}{cc}
        \textbf{M}             & \textbf{F} \\
        \textbf{F}^{\intercal} & \textbf{D}
    \end{array}\right] 
    \left(\begin{array}{c}
        u \\
        \lambda
    \end{array}\right) = 
    \left(\begin{array}{c}
        \bar{g} \\
        \bar{g}_\delta
    \end{array}\right),
\end{equation}
\noindent
and following from {\color{red} [cite]} we use the Schur complement, $
(\textbf{D} - \textbf{F}^{\intercal} \textbf{M}^{-1} \textbf{F}) \lambda 
= \bar{g}_\delta - \textbf{F}^{\intercal} \textbf{M}^{-1} \bar{g}$, to 
solve this system in two parts: the global problem,
\begin{subequations}
\begin{equation}
\lambda_{\textbf{A}} := \bar{g}_\delta - \textbf{F}^{\intercal} \text{solve}(\textbf{M}, \bar{g})
\end{equation}
\begin{equation}
\lambda_{\textbf{b}} := \textbf{D} - \textbf{F}^{\intercal} \text{solve}(\textbf{M}, \textbf{F})
\end{equation}
\begin{equation}
\lambda := \text{solve}(\lambda_{\textbf{A}}, \lambda_{\textbf{b}}),
\end{equation}
\end{subequations}
\noindent
and the local problem,
\begin{equation} 
u := \text{solve}(\textbf{M}, (\bar{g} - \textbf{F} \times \lambda)).
\end{equation}


\begin{tikzpicture}[scale=1.6]
\filldraw[color=black!10] (0, 0) rectangle (3, 3);
\filldraw[color=black!30, text=black] (0.00cm, 2.66cm) rectangle (0.33cm, 3.00cm) node[pos=.5] {$\textbf{M}_1$};
\filldraw[color=black!30, text=black] (0.33cm, 2.33cm) rectangle (0.66cm, 2.66cm) node[pos=.5] {$\textbf{M}_2$};
\filldraw[color=black!30, text=black] (0.66cm, 2.00cm) rectangle (1.00cm, 2.33cm) node[pos=.5] {$\textbf{M}_3$};
\filldraw[color=black!30, text=black] (1.00cm, 1.66cm) rectangle (1.33cm, 2.00cm) node[pos=.5] {$\textbf{M}_4$};
\filldraw[color=black!30, text=black] (1.33cm, 1.33cm) rectangle (1.66cm, 1.66cm) node[pos=.5] {$\textbf{M}_5$};
\filldraw[color=black!30, text=black] (1.66cm, 1.00cm) rectangle (2.00cm, 1.33cm) node[pos=.5] {$\textbf{M}_6$};
\filldraw[color=black!30, text=black] (2.00cm, 0.66cm) rectangle (2.33cm, 1.00cm) node[pos=.5] {$\textbf{M}_7$};
\filldraw[color=black!30, text=black] (2.33cm, 0.33cm) rectangle (2.66cm, 0.66cm) node[pos=.5] {$\textbf{M}_8$};
\filldraw[color=black!30, text=black] (2.66cm, 0.00cm) rectangle (3.00cm, 0.33cm) node[pos=.5] {$\textbf{M}_9$};

\filldraw[color=black!10,xshift=0.1cm] (3, 0) rectangle (4.33, 3);

\filldraw[color=red!30,xshift=0.1cm] (3.00cm, 2.66cm) rectangle (3.11cm, 3.00cm);
\filldraw[color=red!30,xshift=0.1cm] (3.11cm, 2.33cm) rectangle (3.22cm, 2.66cm);
\filldraw[color=red!30,xshift=0.1cm] (3.22cm, 1.66cm) rectangle (3.33cm, 2.00cm);
\filldraw[color=red!30,xshift=0.1cm] (3.33cm, 1.33cm) rectangle (3.44cm, 1.66cm);
\filldraw[color=red!30,xshift=0.1cm] (3.44cm, 0.66cm) rectangle (3.55cm, 1.00cm);
\filldraw[color=red!30,xshift=0.1cm] (3.55cm, 0.33cm) rectangle (3.66cm, 0.66cm);

\filldraw[color=green!30,xshift=0.1cm] (3.00cm, 2.33cm) rectangle (3.11cm, 2.66cm);
\filldraw[color=green!30,xshift=0.1cm] (3.11cm, 2.00cm) rectangle (3.22cm, 2.33cm);
\filldraw[color=green!30,xshift=0.1cm] (3.22cm, 1.33cm) rectangle (3.33cm, 1.66cm);
\filldraw[color=green!30,xshift=0.1cm] (3.33cm, 1.00cm) rectangle (3.44cm, 1.33cm);
\filldraw[color=green!30,xshift=0.1cm] (3.44cm, 0.33cm) rectangle (3.55cm, 0.66cm);
\filldraw[color=green!30,xshift=0.1cm] (3.55cm, 0.00cm) rectangle (3.66cm, 0.33cm);

\filldraw[color=violet!30,xshift=0.1cm] (3.66cm, 2.66cm) rectangle (3.77cm, 3.00cm);
\filldraw[color=violet!30,xshift=0.1cm] (3.77cm, 2.33cm) rectangle (3.88cm, 2.66cm);
\filldraw[color=violet!30,xshift=0.1cm] (3.88cm, 2.00cm) rectangle (4.00cm, 2.33cm);
\filldraw[color=violet!30,xshift=0.1cm] (4.00cm, 1.66cm) rectangle (4.11cm, 2.00cm);
\filldraw[color=violet!30,xshift=0.1cm] (4.11cm, 1.33cm) rectangle (4.22cm, 1.66cm);
\filldraw[color=violet!30,xshift=0.1cm] (4.22cm, 1.00cm) rectangle (4.33cm, 1.33cm);

\filldraw[color=orange!30,xshift=0.1cm] (3.66cm, 1.66cm) rectangle (3.77cm, 2.00cm);
\filldraw[color=orange!30,xshift=0.1cm] (3.77cm, 1.33cm) rectangle (3.88cm, 1.66cm);
\filldraw[color=orange!30,xshift=0.1cm] (3.88cm, 1.00cm) rectangle (4.00cm, 1.33cm);
\filldraw[color=orange!30,xshift=0.1cm] (4.00cm, 0.66cm) rectangle (4.11cm, 1.00cm);
\filldraw[color=orange!30,xshift=0.1cm] (4.11cm, 0.33cm) rectangle (4.22cm, 0.66cm);
\filldraw[color=orange!30,xshift=0.1cm] (4.22cm, 0.00cm) rectangle (4.33cm, 0.33cm);

\filldraw[color=black!10,yshift=-0.1cm] (0, -1.33) rectangle (3, 0);

\begin{scope}[rotate around={-90:(0,0)},yscale=1,xscale=-1,xshift=-3.86cm,yshift=-0.33cm]
 
\filldraw[color=red!30,xshift=0.1cm] (3.00cm, 2.66cm) rectangle (3.11cm, 3.00cm);
\filldraw[color=red!30,xshift=0.1cm] (3.11cm, 2.33cm) rectangle (3.22cm, 2.66cm);
\filldraw[color=red!30,xshift=0.1cm] (3.22cm, 1.66cm) rectangle (3.33cm, 2.00cm);
\filldraw[color=red!30,xshift=0.1cm] (3.33cm, 1.33cm) rectangle (3.44cm, 1.66cm);
\filldraw[color=red!30,xshift=0.1cm] (3.44cm, 0.66cm) rectangle (3.55cm, 1.00cm);
\filldraw[color=red!30,xshift=0.1cm] (3.55cm, 0.33cm) rectangle (3.66cm, 0.66cm);

\end{scope}
\begin{scope}[rotate around={-90:(0,0)},yscale=1,xscale=-1,xshift=-3.86cm,yshift=0.34cm]

\filldraw[color=green!30,xshift=0.1cm] (3.00cm, 2.33cm) rectangle (3.11cm, 2.66cm);
\filldraw[color=green!30,xshift=0.1cm] (3.11cm, 2.00cm) rectangle (3.22cm, 2.33cm);
\filldraw[color=green!30,xshift=0.1cm] (3.22cm, 1.33cm) rectangle (3.33cm, 1.66cm);
\filldraw[color=green!30,xshift=0.1cm] (3.33cm, 1.00cm) rectangle (3.44cm, 1.33cm);
\filldraw[color=green!30,xshift=0.1cm] (3.44cm, 0.33cm) rectangle (3.55cm, 0.66cm);
\filldraw[color=green!30,xshift=0.1cm] (3.55cm, 0.00cm) rectangle (3.66cm, 0.33cm);

\end{scope}
\begin{scope}[rotate around={-90:(0,0)},yscale=1,xscale=-1,xshift=-5.19cm,yshift=1cm] 

\filldraw[color=orange!30,xshift=0.1cm] (3.66cm, 1.66cm) rectangle (3.77cm, 2.00cm);
\filldraw[color=orange!30,xshift=0.1cm] (3.77cm, 1.33cm) rectangle (3.88cm, 1.66cm);
\filldraw[color=orange!30,xshift=0.1cm] (3.88cm, 1.00cm) rectangle (4.00cm, 1.33cm);
\filldraw[color=orange!30,xshift=0.1cm] (4.00cm, 0.66cm) rectangle (4.11cm, 1.00cm);
\filldraw[color=orange!30,xshift=0.1cm] (4.11cm, 0.33cm) rectangle (4.22cm, 0.66cm);
\filldraw[color=orange!30,xshift=0.1cm] (4.22cm, 0.00cm) rectangle (4.33cm, 0.33cm);

\end{scope}

\begin{scope}[rotate around={-90:(0,0)},yscale=1,xscale=-1,xshift=-5.19cm,yshift=-1cm]

\filldraw[color=violet!30,xshift=0.1cm] (3.66cm, 2.66cm) rectangle (3.77cm, 3.00cm);
\filldraw[color=violet!30,xshift=0.1cm] (3.77cm, 2.33cm) rectangle (3.88cm, 2.66cm);
\filldraw[color=violet!30,xshift=0.1cm] (3.88cm, 2.00cm) rectangle (4.00cm, 2.33cm);
\filldraw[color=violet!30,xshift=0.1cm] (4.00cm, 1.66cm) rectangle (4.11cm, 2.00cm);
\filldraw[color=violet!30,xshift=0.1cm] (4.11cm, 1.33cm) rectangle (4.22cm, 1.66cm);
\filldraw[color=violet!30,xshift=0.1cm] (4.22cm, 1.00cm) rectangle (4.33cm, 1.33cm);

\end{scope}

\filldraw[color=black!10,xshift=0.1cm,yshift=-0.1cm] (3, -1.33) rectangle (4.33, 0);

\draw[color=black!30,xshift=0.1cm,yshift=-0.1cm] (3, 0) -- (4.33, -1.33);

\draw [black, thick] (-0.1,-1.43) to [square left brace] (-0.1,3);

\draw [black, thick] (4.53,-1.43) to [square right brace] (4.53,3);

\filldraw[draw=black,fill=white] (0.39cm, -2.1cm) rectangle (4.09cm, -1.53cm);

\filldraw[color=black!10,xshift=0.49cm, text=black] (0cm, -2cm) rectangle (0.5cm, -1.63cm) node[pos=.5] {$0$};
\filldraw[color=black!30,xshift=1.09cm, text=black] (0cm, -2cm) rectangle (0.5cm, -1.63cm) node[pos=.5] {$\textbf{M}_i$};
\filldraw[color=red!30,xshift=1.69cm, text=black] (0cm, -2cm) rectangle (0.5cm, -1.63cm) node[pos=.5] {$\textbf{F}^1$};
\filldraw[color=green!30,xshift=2.29cm, text=black] (0cm, -2cm) rectangle (0.5cm, -1.63cm) node[pos=.5] {$\textbf{F}^2$};
\filldraw[color=violet!30,xshift=2.89cm, text=black] (0cm, -2cm) rectangle (0.5cm, -1.63cm) node[pos=.5] {$\textbf{F}^3$};
\filldraw[color=orange!30,xshift=3.49cm, text=black] (0cm, -2cm) rectangle (0.5cm, -1.63cm) node[pos=.5] {$\textbf{F}^4$};
%\draw [decorate,decoration={brace,amplitude=10pt},xshift=-2pt,yshift=0pt]
%(0cm,2cm) -- (0cm,3cm) node [black,midway,xshift=-0.6cm] 
%{\footnotesize $n$};

\end{tikzpicture}




\subsection{Complexity of the trace system}

\noindent
The matrix used to solve $\lambda$, i.e., $\lambda_A = D - F^{\intercal}M^{-1}F$ is $NI \times NI$. Fixing the global row length, $\bar{N}$, the local row length, $N$, is given in terms of the number of elements per row, $N = \frac{\bar{N}}{E}$, and the number of interfaces is given as $I = 2E^2 - 2E$. The dimensions of $\lambda_A$ can then be written as 
\begin{equation}
	2\bar{N}E - 2\bar{N} \times 2\bar{N}E - 2\bar{N}. 
\end{equation}
The number of non-zeros in $F^{\intercal}M^{-1}F$ are given by
\begin{subequations}
\begin{equation}
	N \sum_{i=1}^{E^2} (\phi_i^2)
\end{equation}
where
\begin{equation}
	\textbf{F}^{\text{ind}} = \left[\textbf{f}^{\text{ind}}_1 \cdots \textbf{f}^{\text{ind}}_{E^2}\right] \in \mathbb{Z}^{E^2 \times I},
\end{equation}
\begin{equation}
	\phi = \left[ \text{nnz}(\textbf{f}^{\text{ind}}_1) \cdots \text{nnz}(\textbf{f}^{\text{ind}}_{E^2}) \right] \in \mathbb{Z}^{E^2}, 
\end{equation}
\end{subequations}
or, the sum of the square of the number of internal interfaces per element. As this problem constrains the layout of element we can write $\phi$ as 
\begin{equation}
	\phi = \left[ \underbrace{2, 2, 2, 2}_{4}, \underbrace{3 \cdots 3}_{4E - 8}, \underbrace{4 \cdots 4}_{(E-2)^2} \right] \in \mathbb{Z}^{E^2}, 
\end{equation}
where $2\text{'s}$ denote the corner elements, $3\text{'s}$ denote the boundary elements, and $4\text{'s}$ denote the internal elements. The ordering of the rows in $\textbf{F}^{\text{ind}}$ may not be equivalent in this case, but the final sum will be identical. Throughout the rest of the problem this is not an issue if the ordering remains consistent. 

With this, we derive the non-zero density of $\textbf{F}^{\intercal}\textbf{M}^{-1}\textbf{F}$ as a function of the number of elements per row 
\begin{equation}
	\text{nzd}(E) = \frac{\bar{N}/E (16E^2 - 28E + 8)}{(2\bar{N}E - 2\bar{N})^2}.
\end{equation}
\begin{equation}
	\text{nzd}(E) = \frac{16E^2 - 28E + 8}{4E^3 - 4E^2 + E}.
\end{equation}

\begin{center}
\pgfplotsset{compat=newest, width=\columnwidth}
\begin{tikzpicture}
\begin{axis}[height=2in,
	xlabel={$E^2$},
    ylabel={Non-zero density}]
    \addplot[color=red,smooth,thick] table[col sep=comma,header=false,
      x index=0,y index=1] {data/density};
\end{axis}
\end{tikzpicture}
\end{center}
% 10000, 50000, 100000, 1000000



\noindent
We store $\textbf{F}$ as a vector of vectors to compute $\textbf{M}^{-1}\textbf{F}$. That is, for each directional $\textbf{F}^k$ boundary we store a list of $N$ vectors

\begin{equation}
	\textbf{F}^k = \left[\textbf{f}^k_1 \cdots \textbf{f}^k_N\right] \in \mathbb{R}^{N \times N^2}.
\end{equation} 

\noindent
In our implementation each vector $\textbf{f}^k_i$ is a PETSc vector. This method is in contrast to storing $\textbf{F}$ contiguously, and allows for the computation of $\textbf{M}^{-1}\textbf{F}$ from the local matrices of $\textbf{M}$, stored similarly as

\begin{equation}
	\textbf{M} = \left\{\textbf{M}_1 \cdots \textbf{M}_p\right\} \in \mathbb{R}^{N_{E}N^2 \times N_{E}N^2}, p \leq E.
\end{equation} 

\noindent 
Where $p$ denotes the number of unique local matrix operators in $\textbf{M}$. 

With this we compute the intermediate result $\textbf{M}^{-1}\textbf{F}$, performing a linear solve of every unique local matrix operator, and every vector of every directional boundary coefficient matrix 

\begin{equation}
	\text{solve}(\textbf{M}_{i}, \textbf{F}^{k}_j) \text{  for }
	\begin{array}{l}
		0 < i \leq p, \\
		0 < j \leq N, \\
		0 < k \leq D. \\ 
	\end{array}
\end{equation}

\noindent
The resultant has a similar non-zero pattern as $\textbf{F}$, following the same symbolic block structure. In our implementation we solve this system through direct solvers made available through PETSc. 

\begin{aside}
	In the homogeneous Poisson equation we have assembled in 
	eq {\color{red} ?? } 

	\begin{equation}
		p = \begin{cases}
		    N_{EF} &  0 \leq N_{EF} < 3 \\ 
			3 &  3 \leq N_{EF} \\ 
		\end{cases}.
	\end{equation} 

	\noindent 
	When $p = 3$ we have one unique $\textbf{M}_i$ for the elements 
	along the top Neumann boundaries, the bottom Neumann boundaries, 
	and the interior elements. Since $p$ is constant when 
	$E \geq 3$, computational complexity decreases \emph{weakly} 
	as the number of elements in the system increases. Additionally, 
	if we fix $N^2E^2$ and increase $E$, we decrease the size of 
	each solve.
\end{aside}

\noindent
The resultant $\textbf{M}^{-1}\textbf{F}$ is stored as $p \times D \times N$ PETSc vectors of length $N^2$, i.e.,
\begin{subequations}
\begin{equation}
	\textbf{M}^{-1}\textbf{F} = [\alpha_{111} \cdots \alpha_{ijk} \cdots \alpha_{NDp}] \in \mathbb{R}^{pDN \times N^2},
\end{equation}
\begin{equation}
	\alpha_{ijk} = \text{solve}(\textbf{M}_i, \textbf{f}_{j}^{k}).
\end{equation}
 This is used in the matrix product $(\textbf{F}^{\intercal}) \times (\textbf{M}^{-1}\textbf{F})$, computing
\begin{equation}
	\Phi = N \sum_{i = 1}^{E} (\phi_i^2)
\end{equation}
vector dot products s.t.,
\begin{equation}
	\textbf{F}^{\text{ind}} = \left[\textbf{f}^{\text{ind}}_1 \cdots \textbf{f}^{\text{ind}}_{E}\right] \in \mathbb{Z}^{E \times I},
\end{equation}
\begin{equation}
	\phi = \left[ \text{nnz}(\textbf{f}^{\text{ind}}_1) \cdots \text{nnz}(\textbf{f}^{\text{ind}}_E) \right] \in \mathbb{Z}^{E}.
\end{equation}
Thus we have 
\begin{equation}
	(\textbf{F}^{\intercal}) \times (\textbf{M}^{-1}\textbf{F}) = 
	\left[\begin{array}{c}
		\beta_{1 1 1} \cdots \beta_{\phi_1 \phi_1 N} \in \mathbb{R}^{\phi_1^2N} \\ 
		\vdots \\
		\beta_{1 1 1} \cdots \beta_{\phi_E \phi_E N} \in \mathbb{R}^{\phi_E^2N} 
	\end{array}\right] 
\end{equation}
\begin{equation}
	\beta_{i j k l} = \alpha_{iyl} \textbf{f}^{u}_{l}, \text{ where } y = \text{nzval}(\textbf{f}^{\text{ind}}_{ij}), u = \text{nzval}(\textbf{f}^{\text{ind}}_{ik}),
\end{equation}
\end{subequations}
where $\text{nzval}(v_i)$ returns the $i\text{th}$ non-zero value in the vector $v$. Notably here the result is not square of all non-zeros in $\textbf{F}^\text{ind}$, but the sum of the squares of each row in $\textbf{F}^\text{ind}$, making the storage dependent upon $p$, the number unique elements and the sparsity of $\textbf{F}$. 

The index for any scalar value $\beta_{i j k l}$ in the matrix representation of $\textbf{F}^{\intercal}\textbf{M}^{-1}\textbf{F}$ is given as 
\begin{equation}
[(N - 1) i + \text{nzind}(\textbf{f}^{\text{ind}}_{ij}) + l, (N - 1) i + \text{nzind}(\textbf{f}^{\text{ind}}_{ik}) + l]
\end{equation}
\noindent
where $\text{nzind}(v_i)$ returns the index of the $i\text{th}$ non-zero of the vector $v$. Since 
%\begin{}
