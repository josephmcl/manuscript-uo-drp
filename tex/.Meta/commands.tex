\definecolor{SpecialBlue}{rgb}{0.10,0.58,0.9}
\definecolor{SpecialGreen}{rgb}{0.04,0.66,0.35}
\definecolor{SpecialRed}{rgb}{0.90,0.01,0.23}

\hypersetup{
    colorlinks = true, 
    urlcolor   = SpecialGreen, 
    linkcolor  = SpecialGreen, 
    citecolor  = SpecialGreen}

\newcommand\convertdate[1]{\expandafter\convertdateaux#1\relax}
\def\convertdateaux#1 #2, #3\relax{#2.\@ #1 #3}

% \pretitle{\begin{center}\normalfont\Large {\bfseries\color{SpecialBlue} 
%    {\Huge!!} \kern 1.5 em DRAFT COPY -- Do Distrubte \kern 1.5 em {\Huge!!}\\} }
% \posttitle{\par\end{center}}

%\predate{\begin{center}\normalfont\bfseries\large\color{SpecialBlue}}
% \date{«\kern 0.75em Compiled on \today\kern 0.75em »}
% \postdate{\end{center}}

\pretitle{\Large\justifying\noindent\bfseries}
\posttitle{\\[1em]}

\preauthor{\justifying\noindent}
\postauthor{\\[1em]}

\predate{}
\date{}
\postdate{}

\renewcommand{\abstractnamefont}{\normalfont\bfseries}
\renewcommand{\abstracttextfont}{\normalfont\small\itshape}
\renewcommand{\linenumberfont}{\normalfont\footnotesize\color{SpecialGreen}\emph}

\newcommand{\fakecite}{{\color{SpecialBlue} \guillemotleft\kern 0.5em cite\kern 0.5em\guillemotright}}

\definecolor{rulecolor}{rgb}{0.39,0.36,0.39}

\definecolor{codegreen}{rgb}{0,0.6,0}
\definecolor{codegray}{rgb}{0.5,0.5,0.5}
\definecolor{codepurple}{rgb}{0.58,0,0.82}
\definecolor{backcolour}{rgb}{0.95,0.95,0.92}
\lstdefinestyle{mystyle}{
    backgroundcolor=\color{backcolour},   
    commentstyle=\color{codegreen},
    keywordstyle=\color{magenta},
    numberstyle=\tiny\color{codegray},
    stringstyle=\color{codepurple},
    basicstyle=\ttfamily\footnotesize,
    breakatwhitespace=false,         
    breaklines=true,                 
    captionpos=b,                    
    keepspaces=true,                 
    numbers=left,                    
    numbersep=5pt,                  
    showspaces=false,                
    showstringspaces=false,
    showtabs=false,                  
    tabsize=2
}

\lstset{style=mystyle}

\renewcommand{\abstractname}{}

\renewenvironment{abstract}
 {\small
  \begin{center}
  \bfseries \abstractname \vspace{-3em} \vspace{0em}
  \end{center}
  %{\color{rulecolor}\rule{\linewidth}{1pt}}
  \list{}{%
    \setlength{\leftmargin}{5em}% <---------- CHANGE HERE
    \setlength{\rightmargin}{\leftmargin}%
  }%
  \item\relax}
 {\vspace{-3em}\noindent\color{rulecolor}\rule{\linewidth}{1pt}\vspace{5em}\endlist}



\newenvironment{aside}
  {\begin{mdframed}[style=0,%
    backgroundcolor=black!5,
      bottomline=true,topline=true,leftline=false,rightline=false,leftmargin=1em,rightmargin=1em,%
          innerleftmargin=1em,innerrightmargin=1em,linewidth=1pt,%
      skipabove=7pt,skipbelow=7pt]\small}
  {\end{mdframed}}

\def\mytypesetter#1{% page 813
  \pgfmathparse{#1/pi}%
  \pgfmathprintnumber{\pgfmathresult}$\pi$%
}

\definecolor{pal1}{RGB}{68, 1, 84}
\definecolor{pal2}{RGB}{68, 57, 131}
\definecolor{pal3}{RGB}{49, 104, 142}
\definecolor{pal4}{RGB}{33, 145, 140}
\definecolor{pal5}{RGB}{53, 183, 121}
\definecolor{pal6}{RGB}{44, 215, 67}
\definecolor{pal7}{RGB}{253, 231, 37}

\newcolumntype{C}[1]{>{\centering\arraybackslash}p{#1}}

\newcommand*{\shifttext}[2]{%
  \settowidth{\@tempdima}{#2}%
  \makebox[\@tempdima]{\hspace*{#1}#2}%
}

\newenvironment{mcfigure}
  {\par\medskip\noindent\minipage{\linewidth}}
  {\endminipage\par\medskip}


\tikzset{
    ncbar angle/.initial=90,
    ncbar/.style={
        to path=(\tikztostart)
        -- ($(\tikztostart)!#1!\pgfkeysvalueof{/tikz/ncbar angle}:(\tikztotarget)$)
        -- ($(\tikztotarget)!($(\tikztostart)!#1!\pgfkeysvalueof{/tikz/ncbar angle}:(\tikztotarget)$)!\pgfkeysvalueof{/tikz/ncbar angle}:(\tikztostart)$)
        -- (\tikztotarget)
    },
    ncbar/.default=0.5cm,
}

\tikzset{square left brace/.style={ncbar=0.1cm}}
\tikzset{square right brace/.style={ncbar=-0.1cm}}



\newcommand{\gbd}{\bar{\textbf{g}}_{\symbf{\delta}}}

\pgfplotsset{select coords between index/.style 2 args={
    x filter/.code={
        \ifnum\coordindex<#1\def\pgfmathresult{}\fi
        \ifnum\coordindex>#2\def\pgfmathresult{}\fi
    }
}}


\pgfplotstableset{
    /color cells/min/.initial=0,
    /color cells/max/.initial=1000,
    /color cells/textcolor/.initial=,
    %
    % Usage: 'color cells={min=<value which is mapped to lowest color>, 
    %   max = <value which is mapped to largest>}
    color cells/.code={%
        \pgfqkeys{/color cells}{#1}%
        \pgfkeysalso{%
            postproc cell content/.code={%
                \begingroup
                % acquire the value before any number printer changed it:
                \pgfkeysgetvalue{/pgfplots/table/@preprocessed cell content}\value
                \ifx\value\empty
                    \endgroup
                \else
                \pgfkeys{/pgf/fpu=true}%
                %\pgfmathfloatparsenumber{\value}%
                \pgfmathparse{log10(\value)}%
                \pgfmathfloattofixed{\pgfmathresult}%
                \let\value=\pgfmathresult
                \pgfkeys{/pgf/fpu=false}% this here was the problem
                % map that value:
                \pgfplotscolormapaccess
                    [\pgfkeysvalueof{/color cells/min}:\pgfkeysvalueof{/color cells/max}]
                    {\value}
                    {\pgfkeysvalueof{/pgfplots/colormap name}}%
                % now, \pgfmathresult contains {<R>,<G>,<B>}
                % 
                % acquire the value AFTER any preprocessor or
                % typesetter (like number printer) worked on it:
                \pgfkeysgetvalue{/pgfplots/table/@cell content}\typesetvalue
                \pgfkeysgetvalue{/color cells/textcolor}\textcolorvalue
                % tex-expansion control
                \toks0=\expandafter{\typesetvalue}%
                \xdef\temp{%
                    \noexpand\pgfkeysalso{%
                        @cell content={%
                            \noexpand\cellcolor[rgb]{\pgfmathresult}%
                            \noexpand\definecolor{mapped color}{rgb}{\pgfmathresult}%
                            \ifx\textcolorvalue\empty
                            \else
                                \noexpand\color{\textcolorvalue}%
                            \fi
                            \the\toks0 %
                        }%
                    }%
                }%
                \endgroup
                \temp
                \fi
            }%
        }%
    }
}

\pgfmathdeclarefunction{lg10}{1}{%
    \pgfmathparse{ln(#1)/ln(10)}%
}

\mathtoolsset{showonlyrefs}
